\chapter{Coordinate polari}
\section{Definizioni}
\label{Cha:CoordinatePolariTrig}
\begin{figure} %[htbp]
	\centering
	\includestandalone[width=.6\textwidth]{terzo/polari/polari1}
	\caption{Sistema di riferimento polare}\label{fig:coodpolarpolari}
\end{figure}
\begin{definizionet}{Coordinate polari}{}
Dato un polo\index{Polo} $P$ e un semiretta di origine $P$ chiamata asse polare\index{Asse polare}, diremo coordinate polari\index{Coordinate!polari} la coppia ordinata $(r,\theta)$, dove $r$ è la distanza dal polo $P$ del punto $A$ e $\theta$ è l'ampiezza dell'angolo, misurato in senso antiorario, formato dall'asse polare e la semiretta che passa per il polo e il punto come nella~\vref{fig:coodpolarpolari}. Il numero $r$ si chiama modulo del punto\index{Coordinate!polari!modulo} $A$, e l'angolo $\theta$ ($\theta\in[0,2\pi[$) argomento\index{Coordinate!polari!argomento} di $A$. 
\end{definizionet}
\section{Da cartesiano a polare}
La figura~\vref{fig:cartpolar} soprappone un sistema di riferimento cartesiano a uno polare. Vi è il problema di convertire le coordinate di un punto dalla forma cartesiana alla forma polare. Quindi 
\[A(x,y)\mapsto A(r,\theta) \]
Dal teorema di Pitagora \[r=\sqrt{x^2+y^2} \]
Se $\theta\in[0,2\pi[$
\[\theta=\begin{cases}
\arctan(\dfrac{y}{x})&\text{se}\ x>0 \ \text{e} \ y\geq0\\[8pt]
\arctan(\dfrac{y}{x})+2\pi&\text{se}\ x>0 \ \text{e} \ y<0\\[8pt]
\arctan(\dfrac{y}{x})+\pi&\text{se}\ x<0\\[8pt]
\dfrac{\pi}{2}&\text{se}\ x=0 \ \text{e} \ y>0\\[8pt]
\dfrac{3}{2}\pi&\text{se}\ x=0 \ \text{e} \ y<0
\end{cases} 
\] 
Se $\theta\in]-\pi,\pi]$
\[\theta=\begin{cases}
\arctan(\dfrac{y}{x})&\text{se}\ x>0 \\[8pt]
\arctan(\dfrac{y}{x})+\pi&\text{se}\ x<0 \ \text{e} \ y\geq0\\[8pt]
\arctan(\dfrac{y}{x})-\pi&\text{se}\ x<0\ \text{e} \ y<0\\[8pt]
\dfrac{\pi}{2}&\text{se}\ x=0 \ \text{e} \ y>0\\[8pt]
-\dfrac{\pi}{2}&\text{se}\ x=0 \ \text{e} \ y<0
\end{cases} 
\]
\begin{figure} %[htbp]
	\centering
	\includestandalone[width=.6\textwidth]{terzo/polari/polari2}
	\caption{Cartesiano -- polare}\label{fig:cartpolar}
\end{figure}
\begin{esempiot}{}{} Consideriamo il punto A di coordinate cartesiane $A(2,5)$. Determinare le coordinate in forma polare di $A$ con $\theta\in[0,2\pi[$
	\begin{align*}
	r=&\sqrt{x^2+y^2}\\
	=&\sqrt{2^2+5^2}\\
	=&\sqrt{4+25}\\
	=&\sqrt{29}
	\end{align*}
	\begin{align*}
	\theta=&\arctan\left(\dfrac{x}{y}\right)\\
		 =&\arctan\left(\dfrac{5}{2}\right)\\
	=&1.19029
	\end{align*}
	\[A(\sqrt{29},1.19029) \]
\end{esempiot}
\begin{esempiot}{}{} Consideriamo il punto A di coordinate cartesiane $A(-2,5)$. Determinare le coordinate in forma polare di $A$ con $\theta\in[0,2\pi[$
	\begin{align*}
	r=&\sqrt{x^2+y^2}\\
	=&\sqrt{(-2)^2+5^2}\\
	=&\sqrt{4+25}\\
	=&\sqrt{29}
	\end{align*}
	\begin{align*}
	\theta=&\arctan\left(\dfrac{x}{y}\right)+\pi\\
	=&\arctan\left(-\dfrac{5}{2}\right)+pi\\
	=&1.95130
	\end{align*}
	\[A(\sqrt{29},1.95130) \]
\end{esempiot}
\begin{esempiot}{}{} Consideriamo il punto A di coordinate cartesiane $A(-2,-5)$. Determinare le coordinate in forma polare di $A$ con $\theta\in[0,2\pi[$
	\begin{align*}
	r=&\sqrt{x^2+y^2}\\
	=&\sqrt{(-2)^2+(-5)^2}\\
	=&\sqrt{4+25}\\
	=&\sqrt{29}
	\end{align*}
	\begin{align*}
	\theta=&\arctan\left(\dfrac{x}{y}\right)+\pi\\
	=&\arctan\left(\dfrac{5}{2}\right)+pi\\
	=&4.33188
	\end{align*}
\[A(\sqrt{29},4.33188) \]
\end{esempiot}
\begin{esempiot}{}{} Consideriamo il punto A di coordinate cartesiane $A(2,-5)$. Determinare le coordinate in forma polare di $A$ con $\theta\in[0,2\pi[$
	\begin{align*}
	r=&\sqrt{x^2+y^2}\\
	=&\sqrt{2^2+(-5)^2}\\
	=&\sqrt{4+25}\\
	=&\sqrt{29}
	\end{align*}
	\begin{align*}
	\theta=&\arctan\left(\dfrac{x}{y}\right)+2\pi\\
	=&\arctan\left(-\dfrac{5}{2}\right)+2pi\\
	=&5.09290
	\end{align*}
\end{esempiot}
\begin{esempiot}{}{} Consideriamo il punto A di coordinate cartesiane $A(2,0)$. Determinare le coordinate in forma polare di $A$ con $\theta\in[0,2\pi[$
	\begin{align*}
	r=&\sqrt{x^2+y^2}\\
	=&\sqrt{2^2}\\
	=&\sqrt{4}\\
	=&2
	\end{align*}
	\begin{align*}
	\theta=&\arctan\left(\dfrac{x}{y}\right)\\
	=&\arctan\left(-\dfrac{0}{2}\right)\\
	=&0
	\end{align*}
\end{esempiot}
\begin{esempiot}{}{} Consideriamo il punto A di coordinate cartesiane $A(-2,0)$. Determinare le coordinate in forma polare di $A$ con $\theta\in[0,2\pi[$
	\begin{align*}
	r=&\sqrt{x^2+y^2}\\
	=&\sqrt{(-2)^2}\\
	=&\sqrt{4}\\
	=&2
	\end{align*}
	\begin{align*}
	\theta=&\arctan\left(\dfrac{x}{y}\right)\\
	=&\pi\\
	\end{align*}
\end{esempiot}
\section{Da Polare a cartesiano}
Se $A(r,\theta)$ allora le coordinate cartesiane di $A$ sono\[\begin{cases}
x=r\sin\theta\\
y=r\cos\theta 
\end{cases} \]
\begin{esempiot}{}{} Consideriamo il punto A di coordinate polari\index{Coordinate!polari} $A(2,\dfrac{\pi}{4})$, quali sono le sue coordinate scritte in forma cartesiana?.
	\begin{align*}
	x=&2\sin\dfrac{\pi}{4}\\
	y=&2\cos\dfrac{\pi}{4}\\
	x=&2\dfrac{\sqrt{2}}{2}\\
	y=&2\dfrac{\sqrt{2}}{2}\\
	x=&\sqrt{2}\\
	y=&\sqrt{2}
	\end{align*}
	\[A(\sqrt{2},\sqrt{2}) \]
\end{esempiot}
\begin{esempiot}{}{} Consideriamo il punto A di coordinate polari\index{Coordinate!polari} $A(5,\dfrac{5}{6})\pi$, quali sono le sue coordinate scritte in forma cartesiana?.
	\begin{align*}
	x=&5\sin\left(\dfrac{5}{6}\right)\pi\\
	y=&5\cos\left(\dfrac{5}{6}\right)\pi\\
	x=&-5\dfrac{\sqrt{3}}{2}\\
	y=&5\dfrac{1}{2}\\
		\end{align*}
	\[A(-5\dfrac{\sqrt{3}}{2},\dfrac{5}{2}) \]
\end{esempiot}