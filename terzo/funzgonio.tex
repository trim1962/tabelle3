\chapter{Goniometria}
\label{sec:GONIOMETRIA}
\section{Angoli}
\label{sec:gonioang}
\begin{definizionet}{Angolo}{}
Un angolo\index{Angolo} è la parte di piano compreso fra due semirette dette lati. I lati hanno in comune l'origine chiamata vertice.
\end{definizionet}

Due semirette formano due angoli: uno convesso,\index{Angolo!convesso} l'altro concavo\index{Angolo!concavo} come nella\nobs\vref{fig:angconconvposoneg}. 
\begin{figure} %[htbp]
	\centering
\includestandalone[width=.6\textwidth]{terzo/funzgonioTikz/angoliconcaviconvessi}
	\caption{Angoli concavi e convessi, positivi e negativi}\label{fig:angconconvposoneg}
\end{figure}
\begin{figure} %[htbp]
	\centering
\includestandalone[width=0.9\textwidth]{terzo/funzgonioTikz/angolinotevoli}
	\caption{Angoli notevoli}\label{fig:Angolorettoposneggonio}
\end{figure}
\begin{figure}
\includestandalone[width=\textwidth]{terzo/funzgonioTikz/mappe_concettuali_Angoli_1}
%\includestandalone[width=\textwidth]{terzo/funzgonioTikz/mappagomiometricaangolo}
	\caption{Mappa goniometria l'angolo}\label{fig:MappaGonometria1}
\end{figure}
\begin{definizionet}{Angoli positivi e negativi}{}
	Fissato un lato, l'angolo è positivo\index{Angolo!positivo} se per costruirlo ruotiamo l'altra semiretta in senso antiorario. Un angolo è negativo\index{Angolo!negativo} se ruoteremo l'altro lato in senso orario come nella\nobs\vref{fig:angconconvposoneg}. 
\end{definizionet}
\begin{definizionet}{Angolo giro}{}
Un angolo è giro\index{Angolo!giro} quando dopo la rotazione, le due semirette coincidono. 
\end{definizionet}
\begin{definizionet}{Angolo piatto}{}
Un angolo è piatto\index{Angolo!retto} quando i suoi lati sono sulla stessa retta.
\end{definizionet}
\begin{definizionet}{Angolo retto}{}
Un angolo è retto\index{Angolo!retto} quando è la metà di un angolo piatto. 
\end{definizionet}
\begin{definizionet}{Angolo acuto}{}
Un angolo è acuto\index{Angolo!acuto} se minore di un angolo retto.
\end{definizionet}
\begin{definizionet}{Angolo ottuso}{}
Un angolo è ottuso\index{Angolo!ottuso} se maggiore di un angolo retto.
\end{definizionet}
 La\nobs\vref{fig:Angolorettoposneggonio} mostra i vari casi.
\section{Misura dell'angolo}
\label{sec:MisuraAngoloGonio}

A ogni angolo viene associata una grandezza detta ampiezza. Vi sono varie unità di misura per l'ampiezza, il grado sessagesimale, il grado decimale e il radiante. 
\subsection{Angolo sessagesimale}
\begin{definizionet}{Grado sessagesimale}{}
Un grado\index{Grado!sessagesimale} è la trecentosessantesima parte in un angolo giro. Il grado si suddivide in minuti e secondi. Il minuto è la sessantesima parte di un grado. Il secondo è la sessantesima parte di un minuto. Il secondo è suddiviso in decimi e centesimi.
\end{definizionet}
 Quindi
 \begin{align*}
\ang{;1;}=&\dfrac{\ang{1}}{60}\\
\ang{;;1}=&\dfrac{\ang{;1;}}{60}=\dfrac{\ang{1}}{3600}
 \end{align*}
\begin{esempiot}{Angoli sessagesimali}{}\index{Grado!sessagesimale}
L'angolo \ang{45;30;20} ha l'ampiezza di \ang{45} gradi \ang{;30;} minuti o primi e \ang{;;20} secondi. L'angolo \ang{30;45;23,7} secondo ha l'ampiezza di \ang{30} gradi \ang{;45;} primi e \ang{;;23} secondi e $7$ decimi.
\end{esempiot}
Un angolo giro\index{Angolo!giro} ha un'ampiezza di \ang{360} gradi. L'angolo piatto\index{Angolo!piatto}, metà di un angolo giro, ha quindi, l'ampiezza di \ang{180} gradi. L'angolo retto\index{Angolo!retto} metà di una angolo piatto, ha un'ampiezza di \ang{90} gradi. Per convertire i gradi in primi si moltiplica i gradi  per $\num{60}$ per ottenere i secondi dai primi si moltiplica per  $\num{60}$ per ottenere i secondi dai gradi si moltiplicano i gradi per  $\num{3600}$. Per convertire i secondi in primi si divide per  $\num{60}$. Per convertire i primi in gradi si dividono i primi per  $\num{60}$. Per passare da seconfi a gradi si divide per  $\num{3600}$.
\begin{figure} %[htbp]
	\centering
	\includestandalone[width=0.9\textwidth]{terzo/funzgonioTikz/gradiprimisecondiconversione}
	\caption{Convertire gradi primi e secondi}\label{fig:Convertiregradiprimisecondi}
\end{figure}
\subsection{Angolo sessadecimale}
\begin{definizionet}{Grado sessadecimale}{}\index{Grado!sessadecimale}
	Un grado\index{Grado!sessadecimale} è la trecentosessantesima parte in un angolo giro. Non ha sottomultipli come il grado sessagesimale.
\end{definizionet}
\begin{esempiot}{Conversione}{}
Convertire in forma decimale un angolo di ampiezza pari a \ang{44;58;48} arrotondato alla \tlungarrotandamento
\end{esempiot}
 Abbiamo un angolo di ampiezza pari a \ang{44;58;48} e vogliamo
 scriverlo in forma decimale\index{Grado!forma decimale}. Dato che $\ang{;1;}=\dfrac{\ang{1}}{60}$ e che $\ang{;;1}=\dfrac{\ang{;1;}}{60}=\dfrac{\ang{1}}{3600}$ avremo
\begin{align*}
\alpha=&\ang{44}+\left(\dfrac{58}{60}\right)^{\si{\degree} }+\left(\dfrac{48}{3600}\right)^{\si{\degree} }\\
=&\ang{44}+\left(\dfrac{58\cdot 60+48}{3600}\right)^{\si{\degree}}\\
=&\ang{44}+\left(\dfrac{3528}{3600}\right)^{\si{\degree}}\approx\ang[round-precision=\lungarrotandamento,round-mode=places]{44,98}
\end{align*}
usando la calcolatrice

\begin{center}
	\begin{tabular}{ll}
		\tasto{58}\tastoper\tasto{60}\tastouguale & 3480 \\ 
		\tastoans\tastopiu\tasto{48}\tastouguale & 3528 \\
		\tastoans\tastodiv\tasto{3600}\tastouguale & \num[round-precision=\lungarrotandamento,round-mode=places]{0.98} \\
		\tastoans\tastopiu\tasto{44}\tastouguale&\num[round-precision=\lungarrotandamento,round-mode=places]{44.98} \\
	\end{tabular}
\end{center} 
\begin{esempiot}{Convertire in forma sessagesimale}{}
Convertiamo $7,42^{\circ}$ in gradi minuti e secondi:
\end{esempiot}
\begin{align*}
\alpha^{\si{\degree}}&=\ang{7,42}-\ang{7}=\ang{0.42}\\ 
=&\ang{0.42}\cdot 60=\ang{;25.2;}\\
=&\ang{;25.2;}-\ang{;25;}=\ang{;0.2;}\\
=&\ang{;0.2;}\cdot 60=\ang{;;12}\\
\end{align*}
Quindi \[\alpha=\ang{7;25;12}\]
usando la calcolatrice

\begin{center}
	\begin{tabular}{ll}
		\tasto{7.42}\tastomeno\tasto{7}\tastouguale & \num{0.42} \\ 
		\tastoans\tastoper\tasto{60}\tastouguale & \num{25.2} \\
		\tasto{25.2}\tastomeno\tasto{25}\tastouguale & \num{0.2} \\ 
		\tastoans\tastoper\tasto{60}\tastouguale & \num{12} \\
	\end{tabular}
\end{center} 
\subsection{Radiante}
\begin{definizionet}{Radiante}{}\index{Radiante}
Data una circonferenza di raggio $r$ e un angolo $\alpha$ con il vertice nel centro $C$ della circonferenza, come nella\nobs\vref{fig:radinatidefgonio}, se $l$ è la lunghezza dell'arco di circonferenza sotteso dall'angolo, diciamo radiante il rapporto \[\rho=\dfrac{l}{r} \]
\end{definizionet}
Dato che il radiante è uguale al rapporto fra due lunghezze è segue che è una grandezza adimensionale.
\begin{figure}
	\centering
	\includestandalone[width=0.6\textwidth]{terzo/funzgonioTikz/radianti}
	\caption{Radianti}
	\label{fig:radinatidefgonio}
\end{figure}
Avremo quindi che un angolo ha l'ampiezza di un radiante\index{Radiante} se l'arco di circonferenza $l$ è uguale al raggio $r$.

In un angolo giro\index{Angolo!giro} l'arco è lungo quanto la circonferenza. La sua misura in radianti è quindi:\[\rho=\dfrac{2\pi r}{r}=2\pi\]
Un angolo piatto, meta di un giro, misura \[\rho=\pi\]\index{Angolo!piatto} e un angolo
retto\index{Angolo!retto} misura: \[\rho=\dfrac{\pi}{2} \] 

Per convertire da gradi sessagesimali a radianti\index{Grado!radiante!conversione} si procede in questo modo:
\begin{align*}
\dfrac{l}{2\pi r}&=\dfrac{\alpha}{\ang{360}}\\
\dfrac{\rho}{2\pi}&=\dfrac{\alpha}{\ang{360}}\\
\rho&=\dfrac{\alpha 2\pi}{\ang{360}}\\
\rho&=\dfrac{\pi}{\ang{180}}\alpha
\intertext{segue che per passare da radianti a gradi sessagesimali avremo}
\alpha&=\dfrac{\ang{180}}{\pi}\rho
\end{align*}
Alcuni semplici esempi di conversione fra angoli e radianti
\begin{esempiot}{Radianti in gradi}{}
Quanto corrisponde in gradi un radiante?
\end{esempiot} 
\begin{align*}
\alpha=&\dfrac{180}{\pi}\cdot 1^{\si{\degree} }\\
\approx&\ang[round-precision=\lungarrotandamento,round-mode=places]{57.29577951}\cdot\ang{1}\\
\approx&\ang[round-precision=\lungarrotandamento,round-mode=places]{57.29577951}\\
\end{align*}

 \begin{center}
	\begin{tabular}{ll}
		\tasto{180}\tastodiv\tastopgreco\tastoper\tasto{1}\tastouguale&
		\tasto{\num[round-precision=\lungarrotandamento,round-mode=places]{57.29577951}}\\
	\end{tabular} 
\end{center}
\begin{esempiot}{Gradi in radianti}{}
	Quanto corrisponde in radianti un grado?
\end{esempiot}
 	\begin{align*}
	\rho=&\dfrac{\pi}{180}\cdot 1\\
	\approx&\ang[round-precision=\lungarrotandamento,round-mode=places]{0.017453292}\cdot 1\\
	\approx&\ang[round-precision=\lungarrotandamento,round-mode=places]{0.017453292}\\
	\end{align*}
	 \begin{center}
		\begin{tabular}{ll}
			\tastopgreco\tastodiv\tasto{180}\tastoper\tasto{1}\tastouguale&
			\tasto{\num[round-precision=\lungarrotandamento,round-mode=places]{0.017453292}}\\
		\end{tabular} 
	\end{center}
\section{Funzioni goniometriche}
\label{sec:FunzioniGoniometriche}
\begin{definizionet}{Circonferenza goniometrica}{}
	Dato un sistema di riferimento cartesiano ortogonale, una circonferenza goniometrica\index{Circonferenza!goniometrica} è una circonferenza con centro nell'origine degli assi e raggio uguale a uno. 
\end{definizionet}
 La circonferenza goniometrica incontra gli assi in quattro punti. $A(1,0)$, $B(0,1)$, $C(-1,0)$ e $D(-1,0)$. Costruiamo un angolo $\alpha$ in modo che il suo vertice coincida con il centro della circonferenza e un lato sia la semiretta positiva dell'asse $x$. L'angolo incontra la circonferenza nei punti $A$ e $P$ come nella\nobs\vref{fig:circonferenzagonimetricagonio}. 
\subsection{Coseno}
\label{sec:cosenogonio}
\begin{figure}
	\begin{subfigure}[b]{.5\linewidth}
		\centering
		\includestandalone[width=5cm]{terzo/funzgonioTikz/cosenodefinizione}
		\caption{Coseno definizione}\label{sub:cosenodef}
	\end{subfigure}%
	\begin{subfigure}[b]{.5\linewidth}
	\centering
		\includestandalone[width=0.6\textwidth]{terzo/funzgonioTikz/cosenografico}
		\caption{Coseno grafico}\label{sub:cosenograf}
	\end{subfigure}
	\captionof{figure}{Coseno}
	\label{ztzcos}
\end{figure}
\begin{definizionet}{Coseno}{}
Data la circonferenza goniometrica\nobs\vref{sub:cosenodef}, disegniamo un angolo con centro nell'origine e con un lato coincidente con l'asse delle ascisse. Indichiamo con $\alpha$ la sua ampiezza. L'altro lato dell'angolo incontra la circonferenza in un punto $P$. Diremo coseno\index{Funzione!Coseno!definizione} dell'angolo $\alpha$ e lo indichiamo con $\cos\alpha$, l'ascissa del punto $P$.
\end{definizionet}
\subsection{Andamento coseno}
\label{sec:andamentocoseno}
Al variare dell'ampiezza dell'angolo varia l'ascissa del punto e quindi anche il valore del coseno\index{Funzione!Coseno} dell'angolo. consideriamo la\nobs\vref{fig:AndamentoCoseno1}. Supponiamo di far variare l'angolo $\alpha$ da zero a $\pi$, quindi che $\ang{0}\leq\alpha\leq\ang{180}$ o se utilizziamo i radianti, tra zero e pi greco, $\num{0}\leq\alpha\leq\pi$. 
\begin{description}
	\item[$\alpha_0$] L'angolo ha ampiezza zero. L'ascissa del punto $P_0$ è positiva e vale uno, in questo caso, il coseno di $\alpha$ vale uno.
	\item [$\alpha_1$] L'angolo è compreso fra zero e novanta gradi, quindi fra zero e $\dfrac{\pi}{2} $, il raggio incontra la circonferenza nel punto $P_1$. Il punto ha ascissa positiva quindi il coseno dell'angolo $\alpha_1$ è un numero positivo minore di uno.
	\item [$\alpha_2$] L'angolo è retto. L'ascissa di $P_{2}$ è nulla e quindi $\cos\alpha_2$ è zero. 
	\item [$\alpha_3$] L'angolo è ottuso. La proiezione del punto $P_3$ incontra l'asse $x$ nel semiasse negativo. Quindi $\cos\alpha_3$ è negativo.
	\item [$\alpha_4$] L'angolo è piatto. Il $P_4$ incontra l'asse $x$ nel punto $(-1;0)$. In questo $\cos\alpha_4$ vale meno uno.
\end{description}
\begin{figure}
	\centering
	\includestandalone[width=0.6\textwidth]{terzo/funzgonioTikz/circonferenzagoniometrica}
	\captionof{figure}{Circonferenza goniometrica}
	\label{fig:circonferenzagonimetricagonio}
\end{figure}
Analogo discorso per angoli di ampiezza maggiore di un angolo piatto come nella~\vref{fig:AndamentoCoseno2}.
\begin{description}
	\item [$\alpha_4$] L'angolo è piatto. Il $P_4$ incontra l'asse $x$ nel punto $(-1;0)$. In questo caso il coseno dell'angolo vale meno uno.
	\item [$\alpha_5$] L'angolo è compreso tra \ang{180} e \ang{270} cioè fra $\pi$ e $\dfrac{3}{2}\pi$. L'ascissa del punto $P_5$ è negativa. Quindi $\cos\alpha_2$ è negativo.
	\item [$\alpha_6$] L'angolo è di duecentosettanta gradi $\dfrac{3}{2}\pi$. Il punto ha ascissa zero quindi il coseno è zero.
	\item [$\alpha_7$] L'angolo è compreso tra \ang{270} e \ang{360} quindi compreso tra $\dfrac{3}{2}\pi$ e $\pi$ radianti. Il punto $P_7$ ha ascissa positiva per cui il $\cos\alpha_7$ è positivo.
	\item [$\alpha_8$] L'angolo è di trecentosessanta gradi o $2\pi$ radianti. Il punto ha ascissa uno segue che il coseno vale uno.
\end{description}
Per angoli superiori ad un angolo giro otteniamo gli stessi precedenti risultati. 
Segue che il coseno:
\begin{enumerate}
	\item è limitato e varia fra $-1$ e $+1$ compresi.
	\item è periodico, di periodo pari a \ang{360} o $2\pi$
	\end{enumerate} 
\begin{figure}
		\centering
\includestandalone[width=0.6\textwidth]{terzo/funzgonioTikz/andamentocoseno1}
		\captionof{figure}{Andamento coseno $\ang{0}<\alpha<\ang{180}$}\label{fig:AndamentoCoseno1}
	\end{figure}%
	\begin{figure}
		\centering
\includestandalone[width=0.6\textwidth]{terzo/funzgonioTikz/andamentocoseno2}
		\captionof{figure}{Andamento coseno $\ang{180}<\alpha<\ang{360} $}\label{fig:AndamentoCoseno2}
\end{figure}
\subsection{Seno}
\label{sec:senogonio}
\begin{figure}
	\begin{subfigure}[b]{.5\linewidth}
%		\centering\includegraphics[scale=0.35]{senoalpha-crop}
		\centering
			\includestandalone[width=5cm]{terzo/funzgonioTikz/senodefinizione}
		\caption{Seno definizione}\label{sub:senodef}
	\end{subfigure}%
	\begin{subfigure}[b]{.5\linewidth}
		\centering
		\includestandalone[width=0.6\textwidth]{terzo/funzgonioTikz/senografico}
		\caption{Seno grafico}\label{sub:senograf}
	\end{subfigure}
	\captionof{figure}{Seno}
	\label{tab:funseno}
\end{figure}
\begin{definizionet}{Seno}{}
	Data una circonferenza goniometrica\nobs\vref{sub:senodef}, disegniamo un angolo con centro nell'origine e con un lato coincidente con l'asse delle ascisse. Indichiamo con $\alpha$ la sua ampiezza. L'altro lato dell'angolo incontra la circonferenza in un punto $P$. Diremo seno\index{Funzione!Seno!definizione} dell'angolo $\alpha$ e lo indichiamo con $\sin\alpha$, l'ordinata del punto $P$
\end{definizionet}
\begin{figure}
	\centering
	\includestandalone[width=0.6\textwidth]{terzo/funzgonioTikz/andamentoseno1}
	\captionof{figure}{Andamento seno $\ang{0}<\alpha<\ang{180}$}\label{fig:AndamentoSeno1}
\end{figure}%
\subsection{Andamento seno}
\label{subs:AndamentoSeno}
Al variare dell'angolo varia la posizione del punto $P$ per ciò il valore del seno\index{Funzione!Seno} cambia. Consideriamo la\nobs\vref{fig:AndamentoSeno1}. Supponiamo di far muovere il punto nella semicirconferenza positiva, l'angolo varia da $\ang{0}\leq\alpha\leq\ang{180}$ o se utilizziamo i radianti $0\leq\alpha\leq\pi$. 
\begin{description}
	\item[$\alpha_0$] L'angolo ha ampiezza zero. L'ordinata del punto è zero quindi il seno di $\alpha$ cioè $\sin\alpha_0$ vale zero.
	\item [$\alpha_1$] L'angolo è compreso fra zero e novanta gradi. In radianti è compreso fra zero e $\dfrac{\pi}{2} $. Il raggio incontra la circonferenza nel punto $P_1$. Questo punto ha ordinata positiva. Il seno dell'angolo $\alpha_1$ è un numero positivo minore di uno.
	\item [$\alpha_2$] L'angolo è un angolo retto. L'ordinata del punto $P_2$ è uno. In questo caso il seno vale uno. 
	\item [$\alpha_3$] Il raggio forma un angolo ottuso. Il punto $P_3$ ha ordinata positiva, quindi $\sin\alpha_2$ è maggiore di zero.
	\item [$\alpha_4$] Abbiamo un angolo piatto. Il $P_4$ incontra l'asse $y$ nel punto $(-1;0)$. In questo $\sin\alpha_4$ vale zero.
\end{description}
Analogo discorso per angoli di ampiezza maggiore di un angolo piatto come nella~\vref{fig:AndamentoSeno2}.
\begin{description}
	\item [$\alpha_4$] L'angolo è piatto. Il punto $P_4$ ha coordinate $(-1;0)$. In questo caso $\sin\alpha_4$ vale zero.
	\item [$\alpha_5$] L'angolo è compreso tra \ang{180} e \ang{270}, in radianti è tra $\pi$ e $\dfrac{3}{2}\pi$. L'ordinata del punto $P_5$ è negativa quindi il seno $\sin\alpha_2$ è negativo.
	\item [$\alpha_6$] L'angolo è la somma di tre angoli retti. Il punto ha ordinata meno uno quindi $\sin\alpha_6=-1$.
	\item [$\alpha_7$] L'angolo è compreso tra \ang{270} e \ang{360}, in radianti tra $\dfrac{3}{2}\pi$ e $2\pi$. L'ordinata del punto $P_7$ è negativa quindi il seno è negativo.
	\item [$\alpha_8$] L'angolo è di trecentosessanta gradi $2\pi$. Il punto ha ordinata zero quindi $\sin\alpha_8=0$.
\end{description}
Per angoli superiori a \ang{360} o $2\pi$ otteniamo gli stessi risultati precedenti. 
Il seno quindi:
\begin{enumerate}
	\item è limitato e varia fra $-1$ e $+1$ compresi.
	\item è periodico, di periodo pari a \ang{360} o $2\pi$
\end{enumerate} 
\begin{figure}
	\centering
	\includestandalone[width=0.6\textwidth]{terzo/funzgonioTikz/andamentoseno2}
	\captionof{figure}{Andamento seno $\ang{180}<\alpha<\ang{360}$}\label{fig:AndamentoSeno2}
\end{figure}%
\begin{figure}
	\begin{subfigure}[b]{.5\linewidth}
		\centering\includestandalone[width=0.6\textwidth]{terzo/funzgonioTikz/segnocoseno}
		\caption{Segno coseno}\label{fig:SegnoCoseno}
	\end{subfigure}%
	\begin{subfigure}[b]{.5\linewidth}
		\centering\includestandalone[width=0.6\textwidth]{terzo/funzgonioTikz/segnoseno}
		\caption{Segno seno}\label{fig:SegnoSeno}
	\end{subfigure}
	\begin{subfigure}[b]{.5\linewidth}
		\centering\includestandalone[width=0.6\textwidth]{terzo/funzgonioTikz/segnotangente}
		\caption{Segno tangente}\label{fig:SegnoTangente}
	\end{subfigure}%
	\begin{subfigure}[b]{.5\linewidth}
		\centering\includestandalone[width=0.6\textwidth]{terzo/funzgonioTikz/segnocotangente}
		\caption{Segno cotangente}\label{fig:SegnoCotangente}
	\end{subfigure}
	\captionof{figure}{Segno funzioni goniometriche}
	\label{tab:segnofunzionigoniometriche}
\end{figure}
\subsection{Tangente}
\begin{definizionet}{Tangente}{}
	Data una circonferenza goniometrica\nobs\vref{fig:TangenteDefinizione}, disegno un angolo con centro nell'origine e di ampiezza $\alpha$. Il prolungamento del lato dell'angolo incontra la tangente alla circonferenza per $(1;0)$ in un punto $T$. Diciamo tangente\index{Funzione!Tangente!definizione} dell'angolo $\alpha$ e lo indichiamo con $\tan\alpha$ l'ordinata del punto $T$.
\end{definizionet}
\label{sec:Tangente}
\begin{figure}
	\begin{subfigure}[b]{.5\linewidth}
		\centering
			\includestandalone[width=5cm]{terzo/funzgonioTikz/tangentedefinizione}
		\caption{Tangente definizione}\label{fig:TangenteDefinizione}
	\end{subfigure}%
	\begin{subfigure}[b]{.5\linewidth}
		\centering\includestandalone[width=0.6\textwidth]{terzo/funzgonioTikz/tangentegrafico}
		\caption{Tangente grafico}\label{fig:TangenteGrafico}
	\end{subfigure}
	\captionof{figure}{Tangente}
	\label{tab:funztg}
\end{figure}
\subsection{Andamento tangente}
\label{sec:AndamentoTangente}
Al variare dell'angolo varia anche il valore della tangente\index{Funzione!Tangente}, consideriamo la\nobs\vref{fig:AndamentoTangente1}. Supponiamo di far variare l'angolo $\alpha$ da zero a centottanta gradi o se utilizziamo i radianti $0\leq\alpha\leq\pi$. 
\begin{description}
	\item[$\alpha_0$] L'angolo ha ampiezza zero. Il raggio incontra la retta tangente in $T_0$. In questo caso la tangente dell'angolo $\alpha$ vale zero.
	\item [$\alpha_1$] L'angolo è compreso fra zero e novanta gradi quindi tra zero e $\dfrac{\pi}{2}$, il prolungamento del raggio incontra la retta nel punto $T_1$. La tangente di $\alpha_1$ è un numero positivo.
	\item [$\alpha_2$] L'angolo è retto. In questo caso il prolungamento del raggio non incontra la parallela all'asse $y$. In questo la tangente non esiste. 
	\item [$\alpha_3$] L'angolo è ottuso. Il prolungamento del raggio incontra le retta in $T_3$. I valori dell'ordinata sono negativi. Quindi la tangente è negativa.
	\item [$\alpha_4$] L'angolo formato è piatto. Il prolungamento incontra la retta tangente nel punto $(-1;0)$. In questo $\tan\alpha_4$ vale zero.
\end{description}
Analogo discorso per angoli di ampiezza maggiore di un angolo piatto come nella~\vref{fig:AndamentoTangente2}.
\begin{description}
	\item [$\alpha_4$] L'angolo è piatto. Il la retta tangente nel punto $(-1;0)$. Quindi la tangente vale zero.
	\item [$\alpha_5$] L'angolo è compreso tra \ang{180} e \ang{270} cioè fra $\pi$ e $\dfrac{3}{2}\pi$. Il prolungamento del raggio incontra la retta tangente nel primo quadrante. Quindi $\tan\alpha_5$ è positivo.
	\item [$\alpha_6$] L'angolo è uguale a tre retti. Il prolungamento del raggio non incontra la retta tangente. In questo caso $\tan\alpha_6$ non esiste. 
	\item [$\alpha_7$] L'angolo è compreso tra \ang{180} e \ang{360} cioè in radianti è compreso tra $\dfrac{3}{2}\pi$ e $2\pi$. Il prolungamento del raggio incontra la retta tangente nel quarto quadrante. Quindi la tangente è negativa.
	\item [$\alpha_8$] L'angolo è di trecentosessanta gradi. Il punto ha ordinata zero quindi la tangente è nulla.
\end{description}
Per angoli superiori a \ang{360}, $2\pi$ otteniamo gli stessi casi illustrati in precedenza. 
Possiamo quindi dire che la tangente:
\begin{enumerate}
	\item è illimitata.
	\item è periodica, di periodo pari a \ang{180} o $\pi$
\end{enumerate} 
\begin{figure}
	\centering
\includestandalone[width=0.6\textwidth]{terzo/funzgonioTikz/tangenteandamento1}
	\captionof{figure}{Andamento tangente $\ang{0}<\alpha<\ang{180}$}\label{fig:AndamentoTangente1}
\end{figure}%
\begin{figure}
	\centering
	\includestandalone[width=0.6\textwidth]{terzo/funzgonioTikz/tangenteandamento2}
\captionof{figure}{Andamento tangente $\ang{180}<\alpha<\ang{360}$}\label{fig:AndamentoTangente2}
\end{figure}%
\subsection{Cotangente}
\label{sec:Cotangente}
\begin{figure}
	\begin{subfigure}[b]{.5\linewidth}
		\centering
		\includestandalone[width=5cm]{terzo/funzgonioTikz/cotangentedefinizione}
	\caption{Cotangente}\label{fig:CotangenteDefinizione}
	\end{subfigure}%
	\begin{subfigure}[b]{.5\linewidth}
		\centering\includestandalone[width=0.6\textwidth]{terzo/funzgonioTikz/cotangentegrafico}
		\caption{Cotangente grafico}\label{fig:CotangenteGrafico}
	\end{subfigure}
	\captionof{figure}{Tangente}\label{tab:funzcotg}
\end{figure}
%\begin{figure}
%	\begin{subfigure}[b]{.5\linewidth}
%		\centering
%		\includestandalone[width=5cm]{terzo/funzgonioTikz/cotangentedefinizione}
%		\caption{Cotangente}\label{fig:CotangenteDefinizione}
%	\end{subfigure}%
%	\begin{subfigure}[b]{.5\linewidth}
%		\centering\includegraphics[scale=0.3]{cotgalphagrafico-crop}
%		\caption{Cotangente grafico}\label{fig:CotangenteGrafico}
%	\end{subfigure}
%	\captionof{figure}{Cotangente}
%	\label{tab:funzcotg}
%\end{figure}
\begin{definizionet}{Cotangente}{}
	Data una circonferenza goniometrica\nobs\vref{fig:CotangenteDefinizione}, disegno un angolo con centro nell'origine e di ampiezza $\alpha$. Un lato dell'angolo incontra la tangente alla circonferenza per $(0;1)$ in un punto $C$. Diremo cotangente\index{Funzione!Cotangente!definizione} dell'angolo $\alpha$ e lo indicheremo con $\cot\alpha$ l'ascissa del punto $C$.
\end{definizionet}
\subsection{Andamento Cotangente}\label{sec:AndamentoCotangente}
Al variare dell'angolo varia anche il valore della cotangente\index{Funzione!Cotangente}, consideriamo la\nobs\vref{fig:AndamentoCotangente1}. Supponiamo di far variare l'angolo $\alpha$ da zero a centottanta gradi, quindi che $\ang{0}\leq\alpha\leq\ang{180}$ o se utilizziamo i radianti $0\leq\alpha\leq\pi$.
\begin{figure}
	\centering
	\includestandalone[width=0.6\textwidth]{terzo/funzgonioTikz/cotangenteandamento1}
	\captionof{figure}{Andamento cotangente $\ang{0}<\alpha<\ang{180}$}\label{fig:AndamentoCotangente1}
\end{figure}% 
\begin{description}
	\item[$\alpha_0$] L'angolo ha ampiezza zero. Il raggio non incontra la retta tangente. In questo caso la cotangente dell'angolo $\alpha$ cioè $\cot 0$ non esiste.
	\item [$\alpha_1$] l'angolo è compreso fra zero e novanta gradi, il prolungamento del raggio incontra la retta nel punto $C_1$. La cotangente di $\alpha_1$ è un numero positivo.
	\item [$\alpha_2$] L'angolo è retto. In questo caso il prolungamento del raggio incontra la parallela all'asse $x$. in $C_2$. In questo $\cot\alpha_2$ vale zero. 
	\item [$\alpha_3$] L'angolo è ottuso. Il prolungamento del raggio incontra le retta in $ C_3$ con valori dell'ascissa negativi. Quindi $\cot\alpha_3$ è negativo.
	\item [$\alpha_4$] L'angolo è piatto. Il prolungamento non incontra la retta tasngente. In questo $\cot\alpha_4$ non esiste.
\end{description}
Analogo discorso per angoli di ampiezza maggiore di un angolo piatto come nella~\vref{fig:AndamentoCotangente2}.
\begin{description}
	\item [$\alpha_4$] L'angolo è piatto. Il prolungamento non incontra la retta tangente. In questo $\cot\alpha_4$ non esiste.
	\item [$\alpha_5$] L'angolo è compreso tra \ang{180} e \ang{270}. Il prolungamento del raggio incontra la retta tangente nel primo quadrante. Quindi $\cot\alpha_5$ è positivo.
	\item [$\alpha_6$] L'angolo è di duecentosettanta gradi. In questo caso il prolungamento del raggio incontra la parallela all'asse $x$. in $ C_6$. In questo $\cot\alpha_6$ vale zero. 
	\item [$\alpha_7$] L'angolo è compreso tra \ang{180} e \ang{360}. Il prolungamento del raggio incontra la retta tangente nel secondo quadrante. Quindi $\cot\alpha_7$ è negativo.
	\item [$\alpha_8$] L'angolo è di trecentosessanta gradi. Il raggio non incontra la retta tangente. In questo caso la cotangente dell'angolo $\alpha_8$ cioè $\cot\alpha_8$ non esiste.
\end{description}
Per angoli superiori a \ang{360} otteniamo gli stessi casi illustrati in precedenza. 
Possiamo quindi dire che cotangente:
\begin{enumerate}
	\item è illimitata.
	\item è periodica, di periodo pari a \ang{180} o $\pi$.
\end{enumerate} 
\begin{figure}
	\centering
	\includestandalone[width=0.6\textwidth]{terzo/funzgonioTikz/cotangenteandamento2}
	\captionof{figure}{Andamento cotangente $\ang{180}<\alpha<\ang{360}$}\label{fig:AndamentoCotangente2}
\end{figure}%
\section{Funzioni goniometriche inverse}
Una funzione goniometrica inversa associa ad un valore numerico un angolo.
\begin{figure}
	\begin{subfigure}[b]{.5\linewidth}
		\centering
	\includestandalone[width=4.5cm]{terzo/funzgonioTikz/acrsengrafico2}
		\caption{Grafico di $\arcsin x$}\label{fig:ArcSenGrafico}
	\end{subfigure}%
	\begin{subfigure}[b]{.5\linewidth}
		\centering
	\includestandalone[width=4.5cm]{terzo/funzgonioTikz/acrcosengrafico2}
		\caption{Grafico di $\arccos x$}\label{fig:ArcCosenGrafico}
	\end{subfigure}
		\begin{subfigure}[b]{\linewidth}
			\centering
	\includestandalone{terzo/funzgonioTikz/arcotangente2}
		\caption{Grafico di $\arctan x$}\label{fig:ArcTangenteGrafico}
		\end{subfigure}
		\begin{subfigure}[b]{\linewidth}
		\centering
		\includestandalone{terzo/funzgonioTikz/arcocotangente2}
		\caption{Grafico di $\arccot x$}\label{fig:ArcCotangenteGrafico}
	\end{subfigure}
			\caption{Funzioni inverse}\label{tab:funzinverse_1}
\end{figure}
\begin{definizionet}{Arcoseno}{}\index{Funzione!Arcoseno!definizione}
	La funzione inversa $y=\arcsin x$ associa a un valore numerico compreso fra $[-1,+1]$ e un angolo appartenente all'intervallo $-\frac{\pi}{2}\leq y\leq\frac{\pi}{2}$ . Il grafico\nobs\vref{fig:ArcSenGrafico} 
\end{definizionet}
Quindi il seno è invertibile come nella\nobs\vref{fig:senomenouno} 
\begin{figure}
	\centering
	\includestandalone{terzo/funzgonioTikz/senomenouno}
	\caption[Arco seno]{Arco seno}
	\label{fig:senomenouno}
\end{figure}
\begin{definizionet}{Arcocoseno}{}\index{Funzione!Arcocoseno!definizione}
	La funzione inversa del coseno è l'arcocoseno $y=\arccos x$, è definita in $[-1,+1]$ e ha come valori in uscita angoli in $0\leq y\leq\pi$ come appare dal grafico\nobs\vref{fig:ArcCosenGrafico} 
\end{definizionet}
Quindi il seno è invertibile come nella\nobs\vref{fig:cosmenouno}
\begin{figure}
	\centering
	\includestandalone{terzo/funzgonioTikz/cosmenouno}
	\caption[Arco coseno]{Arco coseno}
	\label{fig:cosmenouno}
\end{figure}
\begin{definizionet}{Arcotangente}{}\index{Funzione!Arcotangente!definizione}
La funzione inversa della tangente è l'arcotangente $y=\arctan x$, è definita in$\R$ e ha come valori in uscita gli angoli $-\frac{\pi}{2}<y<\frac{\pi}{2}$ come appare dal grafico\nobs\vref{fig:ArcTangenteGrafico}
\end{definizionet}
Quindi la tangente è invertibile come nella\nobs\vref{fig:tanmenouno}
\begin{figure}
	\centering
	\includestandalone{terzo/funzgonioTikz/tanmenouno}
	\caption{Arco tangente}
	\label{fig:tanmenouno}
\end{figure}
\begin{figure}
	\centering
	\includestandalone{terzo/funzgonioTikz/cosmenouno}
	\caption[Arco cotangente]{Arco cotangente}
	\label{fig:cotmenouno}
\end{figure}
\begin{definizionet}{Arcocotangente}{}\index{Funzione!Arcocotangente!definizione}
	La funzione inversa della cotangente è l'arcotangente $y=\arccot x$, è definita in$\R$ e ha come valori in uscita gli angoli $0<y<\pi$ come appare dal grafico\nobs\vref{fig:ArcCotangenteGrafico}
\end{definizionet}
Quindi la cotangente è invertibile come nella\nobs\vref{fig:cotmenouno}
\begin{table}
	\centering
	\begin{tabular}{lCC}
	\toprule
	\multicolumn{1}{l}{Funzione}	&\multicolumn{1}{c}{Dominio} &\multicolumn{1}{c}{Codominio}
	 \\
	 \midrule 
Arcoseno&[-1,+1] & -\frac{\pi}{2}\leq y\leq\frac{\pi}{2} \\[.5cm] 
Arcocoseno&[-1,+1] & 0\leq y\leq\pi \\[.5cm] 
Arcotangente&\R&-\frac{\pi}{2}<y<\frac{\pi}{2}\\[.5cm]
Arcocotangente&\R&0<y<\pi\\
\bottomrule
	\end{tabular} 
	\caption{Funzioni inverse e codominio}\label{tab:Funzioni_inv_Cod}
\end{table}
\section{Relazioni fondamentali}
\label{sec:RelazioniFondamentali}
\begin{table} %[tbp]
	\centering
	\renewcommand{\arraystretch}{2}
	\begin{tabular}{rccccc}
	\toprule
	%\backslashbox{Ottengo}{Noto} & $\sin\alpha$ &$\cos\alpha$&$\tan\alpha$ &$\cot\alpha$ & \multirow{2}{1cm}{$\sin\alpha$ $\cos\alpha$} \\[.5cm]
	& $\sin\alpha$ &$\cos\alpha$&$\tan\alpha$ &$\cot\alpha$ & $\sin\alpha$, $\cos\alpha$ \\[.6cm]
	\midrule
	$\sin\alpha={}$&$\sin\alpha$ & $\pm\sqrt{1-\cos^2\alpha}$ &$\pm\dfrac{\tan\alpha}{\sqrt{1+\tan^2\alpha}}$ &$\pm\dfrac{1}{\sqrt{1+\cot^2\alpha}}$ & \\ [.6cm]
	$\cos\alpha={}$& $\pm\sqrt{1-\sin^2\alpha}$ &$\cos\alpha$ & $\pm\dfrac{1}{\sqrt{1+\tan^2\alpha}}$ &$\pm\dfrac{\cot\alpha}{\sqrt{1+\cot^2\alpha}}$ & \\ [.6cm]
	%\hline
	%\hline
	$\tan\alpha={}$&$\pm\dfrac{\sin\alpha}{\sqrt{1-\sin^2\alpha}}$ &$\pm\dfrac{\sqrt{1-\cos^2\alpha}}{\cos\alpha}$&$\tan\alpha$ & $\dfrac{1}{\cot\alpha}$ &$\dfrac{\sin\alpha}{\cos\alpha}$\\ [.6cm]
	%\hline
	$\cot\alpha={}$&$\pm\dfrac{\sqrt{1-\sin^2\alpha}}{\sin\alpha}$ &$\pm\dfrac{cos\alpha}{\sqrt{1-\cos^2\alpha}}$ &$\dfrac{1}{\tan\alpha}$ &$\cot\alpha$ &$\dfrac{\cos\alpha}{\sin\alpha}$\\[.6cm] 
	\bottomrule
	\end{tabular}
	\caption{Seno Coseno Tangente Cotangente}
	\label{tab:SenoCosenoTangenteCotangente}
\end{table}
\begin{figure}
	\begin{subfigure}[b]{.5\linewidth}
		\centering\includestandalone[width=0.6\textwidth]{terzo/funzgonioTikz/CosenoNotoSeno1}
		\caption{A un valore del seno}\label{fig:CosenoNotoSeno1}
	\end{subfigure}%
	\begin{subfigure}[b]{.5\linewidth}
		\centering\includestandalone[width=0.6\textwidth]{terzo/funzgonioTikz/CosenoNotoSeno2}
		\caption{Corrispondono due punti sulla circonferenza}\label{fig:CosenoNotoSeno2}
	\end{subfigure}
	\begin{subfigure}[b]{.5\linewidth}
		\centering\includestandalone[width=0.6\textwidth]{terzo/funzgonioTikz/CosenoNotoSeno3}
		\caption{A questi punti corrispondono due angoli}\label{fig:CosenoNotoSeno3}
	\end{subfigure}%
	\begin{subfigure}[b]{.5\linewidth}
		\centering\includestandalone[width=0.6\textwidth]{terzo/funzgonioTikz/CosenoNotoSeno4}
		\caption{Da cui ottengo due valori per il coseno}\label{fig:CosenoNotoSeno4}
	\end{subfigure}
	\captionof{figure}{Coseno noto seno}\label{fig:CosenoNotoSenoEs1}
\end{figure}
\renewcommand{\arraystretch}{1}
\subsection{Seno e coseno}
\label{sec:RelazioniFondamentaliSenoCoseno}
\begin{teoremat}{Relazione fondamentale goniometria}{}\index{Relazione!fondamentale!goniometria}\index{Funzione!Seno}\index{Funzione!Coseno}
Dato un angolo $\alpha$ allora vale quanto segue:
\begin{equation*}
\cos^2\alpha+\sin^2\alpha=1\label{eqn:RelazioneFondamentaleTrigonometria1}
\end{equation*}
\end{teoremat}
\begin{figure}
	\centering
	\includestandalone[width=0.6\textwidth]{terzo/funzgonioTikz/RelFondGoniometria}
\caption{Relazione fondamentale goniometria}\label{fig:relFondGonio}
\end{figure}
\begin{proof}
	Consideriamo la\nobs\vref{fig:relFondGonio} per il triangolo $POH$ vale il teorema di Pitagora\index{Teorema!Pitagora} \[b^2+c^2=a^2\] Dato che la circonferenza è una circonferenza goniometrica.
	\begin{align*}
	a&=1\\
	b&=\cos\alpha\\
	c&=\sin\alpha\\
	\intertext{otteniamo}
	cos^2\alpha+\sin^2\alpha=&1 \\
	\end{align*}
\end{proof}
Dalla relazione\nobs\vref{eqn:RelazioneFondamentaleTrigonometria1} è possibile ottenere le seguenti\index{Coseno!noto!seno}\index{Seno!noto!coseno}
\begin{align*}
&\cos^{2}\alpha={}1-\sin^{2}\alpha\\
&\cos\alpha={}\pm\sqrt{1-\sin^{2}\alpha}\\
&\sin^{2}\alpha={}1-\cos^{2}\alpha \\
&\sin\alpha={}\pm\sqrt{1-\cos^{2}\alpha}
\end{align*}
\begin{esempiot}{Trovare il valore del coseno di un angolo noto il seno}{}
Trovare il valore del coseno di un angolo noto il seno. Per esempio poniamo 
\end{esempiot}\index{Coseno!noto!seno}
\begin{align*}
\sin\alpha&{}=\dfrac{3}{5}\\
\cos\alpha&=\pm\sqrt{1-\sin^2\alpha}\\
&=\pm\sqrt{1-\left(\dfrac{3}{5}\right)^2}\\
&=\pm\sqrt{1-\dfrac{9}{25}}\\
&=\pm\sqrt{\dfrac{25-9}{25}}\\
&=\pm\sqrt{\dfrac{16}{25}} \\
&=\pm\dfrac{4}{5}=\num{8e-1} 
\end{align*}
Geometricamente il doppio segno davanti alla radice è spiegabile in questo modo:
\begin{enumerate}
	\item ad un valore del seno $M(0,\dfrac{3}{5})$\nobs\vref{fig:CosenoNotoSeno1}
	\item al punto $M$ corrispondono due punti sulla circonferenza $Q$ e $P$\nobs\vref{fig:CosenoNotoSeno2}
	\item $P$ e $Q$ a questi punti corrispondono due angoli $\alpha$ e $\beta$\nobs\vref{fig:CosenoNotoSeno3}
	\item Da questi angoli otteniamo due valori opposti per il coseno $x_1$ e $x_2$\nobs\vref{fig:CosenoNotoSeno4}
	\end{enumerate}
e quindi il doppio segno.
Analogo ragionamento vale per il coseno basta guardare la\nobs\vref{fig:senoNotoCosenoEs1a}

\begin{figure}
	\begin{subfigure}[b]{.5\linewidth}
		\centering\includestandalone[width=0.6\textwidth]{terzo/funzgonioTikz/senoNotoCoseno1}
	\caption{A un valore del coseno}\label{fig:senoNotoCoseno1}
	\end{subfigure}%
	\begin{subfigure}[b]{.5\linewidth}
		\centering\includestandalone[width=0.6\textwidth]{terzo/funzgonioTikz/senoNotoCoseno2}
		\caption{Corrispondono due punti sulla circonferenza}\label{fig:senoNotoCoseno2}
	\end{subfigure}
	\begin{subfigure}[b]{.5\linewidth}
		\centering\includestandalone[width=0.6\textwidth]{terzo/funzgonioTikz/senoNotoCoseno3}
		\caption{Ottengo due angoli}\label{fig:senoNotoCoseno3}
	\end{subfigure}%
	\begin{subfigure}[b]{.5\linewidth}
		\centering\includestandalone[width=0.6\textwidth]{terzo/funzgonioTikz/senoNotoCoseno4}
		\caption{E due valori opposti per il seno}\label{fig:senoNotoCoseno4}
	\end{subfigure}
	\captionof{figure}{Seno noto coseno}\label{fig:senoNotoCosenoEs1a}
\end{figure}
\subsection{Tangente cotangente}
\label{sec:TangenteCotangente}
\begin{definizionet}{Tangente cotangente}{}
	\index{Funzione!Tangente}\index{Funzione!Cotangente}\index{Funzione!Seno}\index{Funzione!Coseno}
	Dato un angolo $\alpha$ allora vale quanto segue:
\begin{align}
\tan\alpha=&{}\dfrac{\sin\alpha}{\cos\alpha}&\alpha&{}\neq\dfrac{\pi}{2}+k\pi\label{equ:tangente1}\\
\cot\alpha=&{}\dfrac{\cos\alpha}{\sin\alpha}& \alpha&{}\neq k\pi\label{equ:cotangente1}\\
\cot\alpha\tan\alpha=&{}1&\alpha&{}\neq k\pi\quad\alpha{}\neq\dfrac{\pi}{2}+k\pi
\end{align}
\end{definizionet}
\begin{teoremat}{Seno e coseno da tangente cotangente}
Dalle formule\nobs\vrefrange{equ:tangente1}{equ:cotangente1} è possibile ottenere le seguenti relazioni
\begin{align*}
&\cos^{2}\alpha={}\dfrac{1}{1+{\tan}^{2}\alpha} &\alpha&{}\neq\dfrac{\pi}{2}+k\pi\\
&\cos\alpha={}\pm\dfrac{1}{\sqrt{1+{\tan}^{2}\alpha}} &\alpha{}&\neq\dfrac{\pi}{2}+k\pi\\
&\sin^{2}\alpha={}\dfrac{\tan^{2}\alpha}{1+\tan^{2}\alpha}&\alpha{}&\neq\dfrac{\pi}{2}+k\pi\\
&\sin\alpha={}\pm\dfrac{\tan\alpha}{\sqrt{1+\tan^{2}\alpha}}&\alpha{}&\neq\dfrac{\pi}{2}+k\pi\\
&\sin^{2}\alpha={}\dfrac{1}{1+{\cot}^{2}\alpha} &\alpha{}&\neq k\pi\\
&\sin\alpha={}\pm\dfrac{1}{\sqrt{1+{\cot}^{2}\alpha}} &\alpha{}&\neq k\pi\\
&\cos^{2}\alpha={}\dfrac{\cot^{2}\alpha}{1+\cot^{2}\alpha}&\alpha{}&\neq k\pi\\
&\cos\alpha={}\pm\dfrac{\cot\alpha}{\sqrt{1+\cot^{2}\alpha}}&\alpha{}&\neq k\pi\\
\end{align*}
\end{teoremat}\index{Seno!noto!tangente}\index{Coseno!noto!tangente}\index{Seno!noto!cotangente}\index{Coseno!noto!cotangente}
\begin{proof}
	Partendo dalla relazione fondamentale della goniometria
\begin{align*}
\cos^2\alpha+\sin^2\alpha=&1
\intertext{divididendo l'uguaglienza per $\cos^2\alpha$ otteniamo:}
\dfrac{\cos^2\alpha}{\cos^2\alpha}+\dfrac{\sin^2\alpha}{\cos^2\alpha}=&\dfrac{1}{\cos^2\alpha}&\alpha\neq\dfrac{\pi}{2}+k\pi\\
1+\tan^2\alpha=&\dfrac{1}{\cos^2\alpha}\\
\cos^2\alpha=&\dfrac{1}{1+\tan^2\alpha}
\intertext{da cui}
\cos\alpha=&\pm\dfrac{1}{\sqrt{1+\tan^2\alpha}}
\intertext{ma}
\sin\alpha=&\tan\alpha\cos\alpha
\intertext{quindi}
\sin\alpha=&\pm\tan\alpha\dfrac{1}{\sqrt{1+\tan^2\alpha}}\\
\sin\alpha=&\pm\dfrac{\tan\alpha}{\sqrt{1+\tan^2\alpha}}
\end{align*}
Ripartendo dalla relazione fondamentale della goniometria
\begin{align*}
\sin^2\alpha+\cos^2\alpha=&1
\intertext{divididendo l'uguaglienza per $\sin^2\alpha$ otteniamo:}
\dfrac{\sin^2\alpha}{\sin^2\alpha}+\dfrac{\cos^2\alpha}{\sin^2\alpha}=&\dfrac{1}{\sin^2\alpha}&\alpha\neq k\pi\\
1+\cot^2\alpha=&\dfrac{1}{\sin^2\alpha}\\
\sin^2\alpha=&\dfrac{1}{1+\cot^2\alpha}
\intertext{da cui}
\sin\alpha=&\pm\dfrac{1}{\sqrt{1+\cot^2\alpha}}
\intertext{ma}
\cos\alpha=&\cot\alpha\sin\alpha
\intertext{quindi}
\cos\alpha=&\pm\cot\alpha\dfrac{1}{\sqrt{1+\cot^2\alpha}}\\
\cos\alpha=&\pm\dfrac{\cot\alpha}{\sqrt{1+\cot^2\alpha}}
\end{align*}
\end{proof}
\begin{esempiot}{Trovare le funzioni nota una}{}
Supponiamo di conoscere un valore di $\tan\alpha$
\begin{align*}
\tan\alpha=&{}\dfrac{3}{5}\\
\cos\alpha=&{}\pm\dfrac{1}{\sqrt{1+\left(\dfrac{3}{5}\right)^2}}\\
\cos\alpha=&{}\pm\dfrac{1}{\sqrt{1+\dfrac{9}{25}}}\\
\cos\alpha=&{}\pm\dfrac{1}{\sqrt{\dfrac{25+9}{25}}}\\
\cos\alpha=&{}\pm\dfrac{1}{\dfrac{\sqrt{34}}{5}}\\
\cos\alpha=&{}\pm\dfrac{5}{\sqrt{34}}\cdot\dfrac{\sqrt{34}}{\sqrt{34}}=\pm\dfrac{5\sqrt{34}}{34}\\
\sin\alpha=&{}\pm\dfrac{\dfrac{3}{5}}{\sqrt{1+\left(\dfrac{3}{5}\right)^2}}\\
\sin\alpha=&{}\pm\dfrac{\dfrac{3}{5}}{\dfrac{\sqrt{34}}{5}}=\pm\dfrac{3}{5}\cdot\dfrac{5}{\sqrt{34}}=\pm\dfrac{3}{\sqrt{34}}\cdot\dfrac{\sqrt{34}}{\sqrt{34}}=\pm\dfrac{3\sqrt{34}}{34}\simeq\num{5.145e-1}
\end{align*}
\end{esempiot}\index{Seno!noto!tangente}\index{Coseno!noto!tangente}
\begin{figure}
	\begin{subfigure}[b]{.5\linewidth}
		\centering\includestandalone[width=0.6\textwidth]{terzo/funzgonioTikz/SenoCosenoNototangente1}
		\caption{A un valore della tangente}\label{fig:SenoCosenoNototangente1}
	\end{subfigure}%
	\begin{subfigure}[b]{.5\linewidth}
		\centering\includestandalone[width=0.6\textwidth]{terzo/funzgonioTikz/SenoCosenoNototangente2}
		\caption{Corrispondono due punti sulla circonferenza}\label{fig:SenoCosenoNototangente2}
	\end{subfigure}
	\begin{subfigure}[b]{.5\linewidth}
		\centering\includestandalone[width=0.6\textwidth]{terzo/funzgonioTikz/SenoCosenoNototangente3}
		\caption{Ottengo due angoli}\label{fig:SenoCosenoNototangente3}
	\end{subfigure}%
	\begin{subfigure}[b]{.5\linewidth}
		\centering\includestandalone[width=0.6\textwidth]{terzo/funzgonioTikz/SenoCosenoNototangente4}
		\caption{E due valori opposti per il seno e il coseno}\label{fig:SenoCosenoNototangente4}
	\end{subfigure}
	\captionof{figure}{Seno Coseno nota la tangente}\label{fig:senocosenoNototangentEs1}
\end{figure}
\section{Angoli associati}
\label{sec:goniometriaAngoliAssociati}
%\altapriorita{Inserire esempi}
\subsection{Angoli supplementari}
\begin{figure}
	\centering
	\begin{subfigure}[b]{.48\linewidth}
	\centering
	\includestandalone[width=\textwidth]{terzo/funzgonioTikz/angoliassociati1}\caption{Angoli supplementari $\alpha$ $\ang{180}-\alpha$}\label{fig:AngoliAssociatisupplementari}		
 \end{subfigure}
 \begin{subfigure}[b]{.48\linewidth}
 	\centering
 	\includestandalone[width=\textwidth]{terzo/funzgonioTikz/angoliassociati2}
 	\caption{Angoli che differiscono di $\ang{180}$ $\alpha$ e $\ang{180}+\alpha$}
 	\label{fig:AngoliAssociatidiff180}	
 \end{subfigure}
 	\begin{subfigure}[b]{.49\linewidth}
 		\centering
 		\includestandalone[width=\textwidth]{terzo/funzgonioTikz/angoliassociati3}
 		\caption{Angoli esplementari $\alpha$ e $\ang{360}-\alpha$}\label{fig:Angolidif360}
 	\end{subfigure}
 	\begin{subfigure}[b]{.49\linewidth}
 		\centering
 		\includestandalone[width=\textwidth]{terzo/funzgonioTikz/angoliopposti}
 		\caption{Angoli opposti}\label{fig:angoliopposti}
 	\end{subfigure}
\caption{Angoli associati}
	\label{fig:angoliassociati}
\end{figure}
Sono angoli la cui somma\index{Angoli!Somma!$\ang{180}$} è un angolo piatto.  Sono supplementari  gli angoli, $\alpha$, $\ang{180}-\alpha$ la figura corrispondente è la\nobs\vref{fig:AngoliAssociatisupplementari}
Per questi angoli valgono le seguenti relazioni
\begin{align*}
\cos\alpha=&{}-\cos(\ang{180}-\alpha)\\
\sin\alpha=&{}+\sin(\ang{180}-\alpha)\\
\tan\alpha=&{}-\tan(\ang{180}-\alpha)\\
\cot\alpha=&{}-\cot(\ang{180}-\alpha)
\end{align*}
\subsection{Angoli la cui differenza è un angolo piatto}
\label{sub:Dif180}
Gli angoli $\alpha$, $\ang{180}+\alpha$ sono angoli la cui differenza\index{Angoli!Differenza!$\ang{180}$} è $\ang{180}$. Costruiamo la\nobs\vref{fig:AngoliAssociatidiff180}

Per questi angoli valgono le seguenti relazioni
\begin{align*}
\cos\alpha=&{}-\cos(\ang{180}+\alpha)\\
\sin\alpha=&{}-\sin(\ang{180}+\alpha)\\
\tan\alpha=&{}+\tan(\ang{180}+\alpha)\\
\cot\alpha=&{}+\cot(\ang{180}+\alpha)
\end{align*}
\subsection{Angoli esplementari}
Sono angoli la cui somma\index{Angoli!Somma!$\ang{360}$} è $\ang{360}$, $\alpha$, $\ang{360}-\alpha$ la figura corrispondente è la\nobs\vref{fig:Angolidif360}
Per questi angoli valgono le seguenti relazioni
\begin{align*}
\cos\alpha=&{}+\cos(\ang{360}-\alpha)\\
\sin\alpha=&{}-\sin(\ang{360}-\alpha)\\
\tan\alpha=&{}-\tan(\ang{360}-\alpha)\\
\cot\alpha=&{}-\cot(\ang{360}-\alpha)
\end{align*}
\subsection{Angoli opposti}
Sono angoli la cui somma\index{Angoli!Somma!$\ang{0}$} è $\ang{0}$, $\alpha$, $-\alpha$ la figura corrispondente è la\nobs\vref{fig:angoliopposti}
Per questi angoli valgono le seguenti relazioni
\begin{align*}
\cos\alpha=&{}+\cos(-\alpha)\\
\sin\alpha=&{}-\sin(-\alpha)\\
\tan\alpha=&{}-\tan(-\alpha)\\
\cot\alpha=&{}-\cot(-\alpha)
\end{align*}
\subsection{Angoli complementari}
\label{sub:AngoliComp}
\begin{figure}
	\begin{subfigure}[b]{.49\linewidth}
 	\centering
 	\includestandalone[width=\textwidth]{terzo/funzgonioTikz/angolicomplementari1}
 	\caption{Angoli complementari $\alpha$ e $\ang{90}-\alpha$}\label{fig:angolicomplementari1}
 \end{subfigure}
 	\begin{subfigure}[b]{.49\linewidth}
 		\centering
 		\includestandalone[width=\textwidth]{terzo/funzgonioTikz/angolicomplementari2}
 		\caption{Angoli che differiscono di $\ang{90}$, $\alpha$ e $\ang{90}+\alpha$}\label{fig:angolicomplementari2}
 	\end{subfigure}
 \begin{subfigure}[b]{.49\linewidth}
 		\centering
 		\includestandalone[width=\textwidth]{terzo/funzgonioTikz/angolicomplementari3}
 		\caption{Angoli la cui somma è $\ang{270}$, $\alpha$ e $\ang{270}-\alpha$}\label{tab:angolicomplementari3}	
 	\end{subfigure}
	\begin{subfigure}[b]{.49\linewidth}
		\centering
		\includestandalone[width=\textwidth]{terzo/funzgonioTikz/angolicomplementari4}
		\caption{Angoli la cui differenza è $\ang{270}$, $\alpha$ e $\ang{270}+\alpha$}\label{tab:angolicomplementari4}	
	\end{subfigure}
	\caption{Angoli complementari}
	\label{fig:angolicomplementari}
\end{figure}
Sono angoli la cui somma\index{Angoli!Somma!retto} è un angolo retto.  $\alpha$ e $\ang{90}-\alpha$ la figura corrispondente è la\nobs\vref{fig:angolicomplementari1}
%\begin{figure} %[H]
%	\centering
%	\includestandalone[width=8.5cm]{terzo/funzgonioTikz/angolicomplementari1}
%		\caption{Angoli complementari $\alpha$ e $\ang{90}-\alpha$}\label{fig:angolicomplementari1}
%\end{figure}
Per questi angoli valgono le seguenti relazioni
\begin{align*}
\cos\alpha=&{}\sin(\ang{90}-\alpha)\\
\sin\alpha=&{}\cos(\ang{90}-\alpha)\\
\tan\alpha=&{}\cot(\ang{90}-\alpha)\\
\cot\alpha=&{}\tan(\ang{90}-\alpha)
\end{align*}
\subsection{Angoli la cui differenza è un angolo retto}
Sono angoli la cui differenza è $\ang{90}$, $\alpha$ e $\ang{90}+\alpha$ la figura corrispondente è la\nobs\vref{fig:angolicomplementari2}
%\begin{figure} %[H]
%	\centering
%	\includestandalone[width=8.5cm]{terzo/funzgonioTikz/angolicomplementari2}
%\caption{Angoli che differiscono di $\ang{90}$, $\alpha$ e $\ang{90}+\alpha$}\label{fig:angolicomplementari2}
%\end{figure}
Per questi angoli valgono le seguenti relazioni
\begin{align*}
\cos\alpha=&{}+\sin(\ang{90}+\alpha)\\
\sin\alpha=&{}-\cos(\ang{90}+\alpha)\\
\tan\alpha=&{}-\cot(\ang{90}+\alpha)\\
\cot\alpha=&{}-\tan(\ang{90}+\alpha)
\end{align*}
\subsection{Angoli la cui somma è tre angoli retti}
Sono angoli la cui somma\index{Angoli!Somma!$\ang{270}$} è $\ang{270}$, $\alpha$ e $\ang{270}-\alpha$ la figura corrispondente è la\nobs\vref{tab:angolicomplementari3}
%\begin{figure} %[H]
%	\centering
%		\includestandalone[width=8.5cm]{terzo/funzgonioTikz/angolicomplementari3}
%		\caption{Angoli la cui somma è $\ang{270}$, $\alpha$ e $\ang{270}-\alpha$}\label{tab:angolicomplementari3}
%\end{figure}
Per questi angoli valgono le seguenti relazioni
\begin{align*}
\cos\alpha=&{}-\sin(\ang{270}-\alpha)\\
\sin\alpha=&{}-\cos(\ang{270}-\alpha)\\
\tan\alpha=&{}+\cot(\ang{270}-\alpha)\\
\cot\alpha=&{}+\tan(\ang{270}-\alpha)
\end{align*}
\subsection{Angoli la cui differenza è tre angoli retti}
Sono angoli la cui differenza è di tre angoli retti. Come per esempio $\alpha$ e $\ang{270}+\alpha$ la figura corrispondente è la\nobs\vref{tab:angolicomplementari4}
%\begin{figure} %[H]
%	\centering
%		\includestandalone[width=8.5cm]{terzo/funzgonioTikz/angolicomplementari4}
%		\caption{Angoli la cui differenza è $\ang{270}$, $\alpha$ e $\ang{270}+\alpha$}\label{tab:angolicomplementari4}
%\end{figure}
Per questi angoli valgono le seguenti relazioni
\begin{align*}
\cos\alpha=&{}-\sin(\ang{270}+\alpha)\\
\sin\alpha=&{}+\cos(\ang{270}+\alpha)\\
\tan\alpha=&{}-\cot(\ang{270}+\alpha)\\
\cot\alpha=&{}-\tan(\ang{270}+\alpha)
\end{align*}
%\mediapriorita{Manca la tabella 270+$\alpha$}
\begin{table}
\centering
	\footnotesize
	\begin{tabular}{rlllllll}
	\toprule
	$\sin\alpha=$&$\cos(\ang{90}-\alpha)$&$-\cos(\ang{90}+\alpha)$&$\sin(\ang{180}-\alpha)$&$-\sin(\ang{180}+\alpha)$&$-\cos(\ang{270}-\alpha)$&$\cos(\ang{270}+\alpha)$&$-\sin(-\alpha)$\\[.6cm] 
	$\cos\alpha=$&$\sin(\ang{90}-\alpha)$&$\sin(\ang{90}+\alpha)$&$-\cos(\ang{180}-\alpha)$&$-\cos(\ang{180}+\alpha)$&$-\sin(\ang{270}-\alpha)$&$-\sin(\ang{270}+\alpha)$&$\cos(-\alpha)$\\[.6cm] 
	$\tan\alpha=$&$\cot(\ang{90}-\alpha)$&$-\cot(\ang{90}+\alpha)$&$-\tan(\ang{180}-\alpha)$&$\tan(\ang{180}+\alpha)$&$\cot(\ang{270}-\alpha)$&$-\cot(\ang{270}+\alpha)$&$-\tan(-\alpha)$\\[.6cm] 
	$\cot\alpha=$&$\tan(\ang{90}-\alpha)$&$-\tan(\ang{90}+\alpha)$&$-\cot(\ang{180}-\alpha)$&$\cot(\ang{180}+\alpha)$&$\tan(\ang{270}-\alpha)$&$-\tan(\ang{270}+\alpha)$&$-\cot(-\alpha)$\\[.6cm]
	\bottomrule
	\end{tabular}
	\caption{Angoli complementari e supplementari}\label{tab:differnzediangoli}
\end{table}
La tabella\nobs\vref{tab:differnzediangoli} riassume i risultati precedenti.
%\section{Formule di addizione e sottrazione}
%\label{sec:Formulediaddizionesottrazione}
%\altapriorita{Inserire testo}
%\subsection{Coseno della differenza di due angoli}
%\label{sec:cosenodifferenza}
%\begin{figure}
%	\centering
%\includestandalone[width=8.5cm]{terzo/funzgonioTikz/cosenosommadifferenza1}
%	\caption{Differenza di angoli}
%	\label{fig:DifferenzaCosenoAngoli}
%\end{figure}
%\begin{table}
%	\centering
%	$
%	\begin{array}{cc}
%	\toprule
%	\cos(\alpha-\beta)=	&\cos\alpha\cos\beta+\sin\alpha\sin\beta \\ 
%	\cos(\alpha+\beta)=	&\cos\alpha\cos\beta-\sin\alpha\sin\beta \\ 
%	\sin\left(\alpha-\beta\right)={} &\sin\alpha\cos\beta-\cos\alpha\sin\beta\\
%\sin\left(\alpha+\beta\right)={}&\sin\alpha\cos\beta+\cos\alpha\sin\beta\\
%	\bottomrule
%	\end{array}
%	$ 
%	\caption{Seno e Coseno, somma o differenza di angoli}
%	\label{Tab:sommadifferenzacosenoseno}
%\end{table}
\chapter{Equazioni goniometriche elementari}
\label{sec:EquazioniGoniometricheElementari}
Le equazioni goniometriche elementari\index{Equazione!elementare!seno} sono equazioni del tipo
\begin{align*}
	\sin x&=m&\cos x&=n&\tan x&=q
\end{align*}
\section{Equazione elementare seno}
\begin{figure}
	\begin{subfigure}[b]{.5\linewidth}
		\centering
		\includestandalone[width=0.6\textwidth]{terzo/funzgonioTikz/EquaElementareSeno4}
		\captionof{figure}{$\sin x=m$\ $-1<m<0$}\label{fig:EquaElementareSeno4}
	\end{subfigure}%
	\begin{subfigure}[b]{.5\linewidth}
		\centering
		\includestandalone[width=0.6\textwidth]{terzo/funzgonioTikz/EquaElementareCoseno4}
		\captionof{figure}{$\cos x=m$ $-1<m<0$}\label{fig:EquaElementareCoseno4}
	\end{subfigure}
	\captionof{figure}{Equazioni Elementari}\label{fig:EquaElementareSenoCoseno2}
\end{figure}
Il seno di un angolo è una funzione limitata che assume valori  tra meno uno ed uno compresi $(1\leq\sin x\leq 1)$ quindi l'equazione è impossibile per valori di $m$ esterni a tale intervallo. Abbiamo vari casi:
\begin{description}
	\item[$0<m<1$] In questo caso le soluzioni sono due come nella\nobs\vref{fig:EquaElementareSeno1} $x=\alpha+k\ang{360}$ e $x=\ang{180}-\alpha+k\ang{360}$
	\item [$m=+1$] Guardando la\nobs\vref{fig:EquaElementareSeno2} la soluzione è unica, $x=\ang{90}+k\ang{360}$
	\item [$m=-1$] Guardando la\nobs\vref{fig:EquaElementareSeno2} la soluzione è unica, $x=\ang{270}+k\ang{360}$
\end{description} 
\begin{figure}
	\begin{subfigure}[b]{.5\linewidth}
		\centering
			\includestandalone[width=0.6\textwidth]{terzo/funzgonioTikz/EquaElementareSeno1}
			\captionof{figure}{$\sin x=m$\ $0<m<1$}\label{fig:EquaElementareSeno1}
	\end{subfigure}%
	\begin{subfigure}[b]{.5\linewidth}
		\centering
		\includestandalone[width=0.6\textwidth]{terzo/funzgonioTikz/EquaElementareCoseno1}
		\captionof{figure}{$\cos x=m$ $0<m<1$}\label{fig:EquaElementareCoseno1}
	\end{subfigure}
	\captionof{figure}{Equazioni Elementari}\label{fig:EquaElementareSenoCoseno}
\end{figure}
\begin{esempiot}{Equazione goniometrica}{}
Risolvere l'equazione $\sin x =\dfrac{\sqrt{2}}{2}$
\end{esempiot}
\begin{align*}
\intertext{Supponiamo di dover risolvere}
\sin x& =\dfrac{\sqrt{2}}{2}
\intertext{quindi}
m& =\dfrac{\sqrt{2}}{2}
\intertext{In questo caso vale la\nobs\vref{fig:EquaElementareSeno1} m è un valore noto quindi l'angolo è}
\alpha&=\ang{45}
\intertext{le due soluzioni sono}
x&=\ang{45}+k\ang{360}\\
 x&=\ang{180}-\ang{45}+k\ang{360}\\
\end{align*}
\begin{esempiot}{Equazione goniometrica}{}
Risolvere l'equazione $\sin x=-\dfrac{\sqrt{2}}{2}$
\end{esempiot}
\begin{align*}
\intertext{Supponiamo di dover risolvere}
\sin x& =-\dfrac{\sqrt{2}}{2}
\intertext{quindi}
m& =-\dfrac{\sqrt{2}}{2}
\intertext{In questo caso vale la\nobs\vref{fig:EquaElementareSeno4} m è un valore noto quindi l'angolo è}
\alpha&=\ang{45}
\intertext{le due soluzioni sono}
x&=\ang{180}+\ang{45}+k\ang{360}\\
 x&=\ang{360}-\ang{45}+k\ang{360}\\
\end{align*}
% % % % % % % % % % % % % % % % % % % % % % % % % % % % % % % % %
\begin{figure}
	\begin{subfigure}[b]{.5\linewidth}
		\centering
	\includestandalone[width=0.6\textwidth]{terzo/funzgonioTikz/EquaElementareSeno2}
	\captionof{figure}{$\sin x=1$}\label{fig:EquaElementareSeno2}
	\end{subfigure}%
	\begin{subfigure}[b]{.5\linewidth}
		\centering
	\includestandalone[width=0.6\textwidth]{terzo/funzgonioTikz/EquaElementareSeno3}
	\captionof{figure}{$\sin x=-1$}\label{fig:EquaElementareSeno3}
	\end{subfigure}
	\captionof{figure}{Seno casi particolari}\label{fig:EquaElementareSeno3a}
\end{figure}
% % % % % % % % % % % % % % % % % % % % % % % % % % % % % % % % %
\begin{esempiot}{Equazione goniometrica}{}
Risolvere l'equazione $ \sin(x+\ang{30})=\dfrac{\sqrt{2}}{2}$
\end{esempiot}
\begin{align*}
\sin(x+\ang{30})&=\dfrac{\sqrt{2}}{2}
\intertext{Trovo la prima soluzione}
x+\ang{30}&=\ang{45}+k\ang{360}\\
x&=\ang{45}-\ang{30}+k\ang{360}\\
x&=\ang{15}+k\ang{360}
\intertext{La soluzione associata è}
x+\ang{30}&=\ang{180}-\ang{45}+k\ang{360}\\
x&=\ang{180}-\ang{45}-\ang{30}+k\ang{360}\\
x&=\ang{105}+k\ang{360}
\intertext{Le due soluzioni sono}
x&=\ang{15}+k\ang{360}\\
x&=\ang{105}+k\ang{360}
\end{align*}
\begin{esempiot}{Equazione goniometrica}{}
Risolvere l'equazione $\sin(4x+\ang{20})=\sin(3x+\ang{30})$
\end{esempiot}
\begin{align*}
\sin(4x+\ang{20})&=\sin(3x+\ang{30})
\intertext{Uguagliamo gli argomenti e trovo la prima soluzione}
4x+\ang{20}&=3x+\ang{30}+k\ang{360}\\
4x-3x&=\ang{30}-\ang{20}+k\ang{360}\\
x&=\ang{10}+k\ang{360}
\intertext{La soluzione associata è}
4x+\ang{20}&=\ang{180}-(3x+\ang{30})+k\ang{360}\\
4x+\ang{20}&=\ang{180}-3x-\ang{30}+k\ang{360}\\
4x-3x&=\ang{180}-\ang{20}-\ang{30}+k\ang{360}\\
x&=\ang{130}+k\ang{360}
\intertext{Le due soluzioni sono}
x&=\ang{10}+k\ang{360}\\
x&=\ang{130}+k\ang{360}
\end{align*}
\begin{table}
\[
\begin{array}{cccc}
\toprule
Equazione &Soluzione & Soluzione & Trasformazione \\ 
\midrule
\sin\alpha=\sin\beta & \alpha=\beta+k\ang{360} & \alpha=\ang{180}-\beta+k\ang{360} &\\
\midrule
\sin\alpha=-\sin\beta&\alpha=-\beta+k\ang{360}&\alpha=\ang{180}+\beta+k\ang{360}&-\sin\beta=\sin(-\beta)\\
%&  &  &\\
\bottomrule
\end{array}
\] 
\caption{Equazioni elementari in seno}
\label{tab:EquazioniElementarinSeno}
\end{table}
L'esempio che segue è quasi analogo al precedente,  prima procedere trasformiamo il secondo membro utilizzando gli angoli associati\nobs\vref{sub:Dif180}. 
\begin{esempiot}{Equazione goniometrica}{}
Risolvere l'equazione $ \sin(3x-\ang{10})=-\sin(5x+\ang{40})$
\end{esempiot}
\begin{align*}
\sin(3x-\ang{10})&=-\sin(5x+\ang{40})
\intertext{Consideriamo l'angolo associato}
\sin(\ang{180}+\alpha)&=-\sin(\alpha)\\
\sin(\ang{180}+5x+\ang{40})&=-\sin(5x+\ang{40})\\
\intertext{L'equazione di partenza diventa}
\sin(3x-\ang{10})&=\sin(\ang{180}+5x+\ang{40})
\intertext{Uguagliamo gli argomenti e trovo la prima soluzione}
3x-\ang{10}&=\ang{180}+5x+\ang{40}+k\ang{360}\\
3x-5x&=\ang{10}+\ang{180}+\ang{40}+k\ang{360}\\
-2x&=\ang{230}+k\ang{360}\\
x&=-\ang{115}+k\ang{180}
\intertext{La soluzione associata è}
3x-\ang{10}&=\ang{180}-(\ang{180}+5x+\ang{40})+k\ang{360}\\
3x-\ang{10}&=-5x-\ang{40}+k\ang{360}\\
3x+5x&=\ang{10}-\ang{40}+k\ang{360}\\
8x&=-\ang{30}+k\ang{360}\\
x&=-\dfrac{\ang{30}}{8}+k\dfrac{\ang{360}}{8}
\intertext{Le due soluzioni sono}
x&=-\ang{115}+k\ang{180}\\
x&=-\dfrac{\ang{30}}{8}+k\ang{45}
\end{align*}
Esempio analogo al precedente. Prima di risolvere l'equazione bisogna trasformare il secondo membro utilizzando gli angoli complementari.\nobs\vref{sub:AngoliComp} 
\begin{esempiot}{Equazione goniometrica}{}
Risolvere l'equazione $\sin(2x+\ang{30})=\cos(x+\ang{40})\\$
\end{esempiot}
	\begin{align*}
\sin(2x+\ang{30})&=\cos(x+\ang{40})\\
\intertext{Consideriamo l'angolo complementare}
\cos(\alpha)&=\sin(90-\alpha)
\intertext{L'equazione di partenza diventa}
\sin(2x+\ang{30})&=\sin([\ang{90}-(x+\ang{40})])+k\ang{360}
\intertext{Uguagliamo gli argomenti e trovo la prima soluzione}
2x+\ang{30}&=\ang{90}-(x+\ang{40})+k\ang{360}\\
2x+x&=\ang{90}-\ang{40}-\ang{30}+k\ang{360}\\
3x&=\ang{20}+k\ang{360}\\
x&=\dfrac{\ang{20}}{3}+k\ang{120}
\intertext{La soluzione associata è}
2x+\ang{30}&=\ang{180}-[\ang{90}-(x+\ang{40})]+k\ang{360}\\
2x+\ang{30}&=\ang{180}-\ang{90}+x+\ang{40}+k\ang{360}\\
2x-x&=\ang{180}-\ang{90}+x+\ang{40}-\ang{30}+k\ang{360}\\
x&=\ang{100}+k\ang{360}
\intertext{Le due soluzioni sono}
x&=\dfrac{\ang{20}}{3}+k\ang{120}\\
x&=\ang{100}+k\ang{360}
	\end{align*}
\section{Equazione elementare coseno}
Il coseno di un angolo\index{Equazione!elementare!coseno} è come il seno, una funzione limitata. Il coseno  assume valori compresi tra meno uno ed uno $(1\leq\cos x\leq 1)$. Quindi l'equazione è impossibile per valori di $m$ esterni a tale intervallo. Abbiamo vari casi:
\begin{description}
	\item[$0<m<1$] Le soluzioni sono due come dalla\nobs\vref{fig:EquaElementareCoseno1} $x=\alpha+k\ang{360}$ e $x=-\alpha+k\ang{360}$
	\item [$m=+1$] La\nobs\vref{fig:EquaElementareCoseno2} mostra la soluzione che è $\alpha=0 + k\ang{360}$
	\item [$m=-1$] La soluzione è $\alpha=\ang{180}+k\ang{360}$ o analogamente $\alpha=-\ang{180}+k\ang{360}$, come segue dalla\nobs\vref{fig:EquaElementareCoseno3} 
\end{description} 
\begin{esempiot}{Equazione goniometrica}{}
Risolvere l'equazione $\cos x =\dfrac{1}{2} $
\end{esempiot}
\begin{align*}
\cos x& =\dfrac{1}{2}
\intertext{quindi}
m& =\dfrac{1}{2}
\intertext{In questo caso vale la\nobs\vref{fig:EquaElementareCoseno1} m è un valore noto quindi l'angolo è}
\alpha&=\ang{60}
\intertext{le due soluzioni sono}
x&=+\ang{60}+k\ang{360}\\
x&=-\ang{60}+k\ang{360}\\
\end{align*}
\begin{figure}
	\begin{subfigure}[b]{.5\linewidth}
		\centering
	\includestandalone[width=0.6\textwidth]{terzo/funzgonioTikz/EquaElementareCoseno3}
	\captionof{figure}{$\cos x=m$ $m=-1$}\label{fig:EquaElementareCoseno3}
	\end{subfigure}%
	\begin{subfigure}[b]{.5\linewidth}
		\centering
		\includestandalone[width=0.6\textwidth]{terzo/funzgonioTikz/EquaElementareCoseno2}
		\captionof{figure}{$\cos x=m$ $m=+1$}\label{fig:EquaElementareCoseno2}
	\end{subfigure}
	\captionof{figure}{Coseno casi particolari}\label{fig:EquaElementareCoseno1a}
\end{figure}
\begin{esempiot}{Equazione goniometrica}{}
Risolvere l'equazione $\cos(5x+\ang{40})=\cos(2x+\ang{30})$
\end{esempiot}
\begin{align*}
\cos(5x+\ang{40})&=\cos(2x+\ang{30})
\intertext{Uguagliamo gli argomenti e trovo la prima soluzione}
5x+\ang{40}&=2x+\ang{30}+k\ang{360}\\
5x-2x&=\ang{30}-\ang{40} +k\ang{360}\\
3x&=-\ang{10} +k\ang{360}\\
x&=-\dfrac{\ang{10}}{3} +k\ang{120}
\intertext{La soluzione associata è}
5x+\ang{40}&=-(2x+\ang{30})+k\ang{360}\\
5x+\ang{40}&=-2x-\ang{30}+k\ang{360}\\
5x+2x&=-\ang{30}-\ang{40}+k\ang{360}\\
7x&=-\ang{70}+k\ang{360}\\
x&=-\ang{10}+k\dfrac{\ang{360}}{7}
\intertext{Le due soluzioni sono}
x&=-\dfrac{\ang{10}}{3} +k\ang{120}\\
x&=-\ang{10}+k\dfrac{\ang{360}}{7}
\end{align*}
\begin{table}
\[
\begin{array}{cccc}
\toprule
Equazione &Soluzione & Soluzione & Trasformazione \\ 
\midrule
\cos\alpha=\cos\beta & \alpha=\beta+k\ang{360} & \alpha=-\beta+k\ang{360} &\\
\midrule
\cos\alpha=-\cos\beta&\alpha=\ang{180}-\beta+k\ang{360}&\alpha=\ang{180}+\beta+k\ang{360}&-\cos\beta=\cos(\ang{180}-\beta)\\
%&  &  &\\
\bottomrule
\end{array}
\] 
\caption{Equazioni elementari in coseno}\label{tab:EquazioniElementariInCoseno}
\end{table}
\begin{esempiot}{Equazione goniometrica}{}
Risolvere l'equazione $\cos(3x-\ang{40})=-\cos(5x-\ang{20})$
\end{esempiot}
	\begin{align*}
\cos(3x-\ang{40})&=-\cos(5x-\ang{20})\\
\cos(\ang{180}-\alpha)&=-\cos(\alpha)\\
\cos(3x-\ang{40})&=\cos(\ang{180}-(5x-\ang{20}))
\intertext{Uguagliamo gli argomenti e trovo la prima soluzione}
3x-\ang{40}&=\ang{180}-(5x-\ang{20})+k\ang{360}\\
3x-\ang{40}&=\ang{180}-5x+\ang{20}+k\ang{360}\\
3x+5x&=\ang{180}-5x+\ang{20}+\ang{40}+k\ang{360}\\
8x&=\ang{240}+k\ang{360}\\
x&=\ang{30}+k\ang{45}
\intertext{La soluzione associata è}
3x-\ang{40}&=-(\ang{180}-(5x-\ang{20}))+k\ang{360}\\
3x-\ang{40}&=-\ang{180}+5x-\ang{20}+k\ang{360}\\
3x-5x&=-\ang{180}-\ang{20}+\ang{40}+k\ang{360}\\
-2x&=-\ang{160}+k\ang{360}\\
x&=\ang{80}+k\ang{180}\\
\intertext{Le due soluzioni sono}
x&=\ang{30}+k\ang{45}\\
x&=\ang{80}+k\ang{180}
	\end{align*}
\section{Equazione elementare tangente}
La tangente\index{Equazione!elementare!tangente} è una funzione non limitata che esiste per valori di $x\neq\ang{90}+k\ang{180}$ 
\begin{figure}
	\centering
	\includestandalone[width=0.6\textwidth]{terzo/funzgonioTikz/EquaElementaretangente1}
	\captionof{figure}{Tangente equazione elementare}\label{fig:EquaElementaretangente1}
\end{figure}%
Se $\alpha$ è soluzione di $\tan x=m$, come si vede dalla\nobs\vref{fig:EquaElementaretangente1} anche $\alpha+\ang{180}$ è soluzione. Dato che la funzione non è sempre definita, la soluzione è accettabile per $x\neq\ang{90}+k\ang{180}$
\begin{esempiot}{Equazione goniometrica}{}
Risolvere l'equazione $\tan x=1$
\end{esempiot}
\begin{align*}
\tan x&=1
\intertext{m è un valore noto quindi l'angolo è}
x&=\ang{45}
\intertext{se \ang{45} è soluzione dalla\nobs\vref{fig:EquaElementaretangente1} anche \ang{45} + \ang{180} è soluzione. Quindi le soluzioni sono:}
x&=\ang{45}+k\ang{180}
	\end{align*}
Un esempio leggermente più complesso. Qui bisogna prima verificare quando esistono le tangenti.
\begin{esempiot}{Equazione goniometrica}{}
Risolvere l'equazione $\tan (3x+\ang{50})=\tan(x-\ang{30})$
\end{esempiot}
\begin{align*}
\intertext{Supponiamo di dover risolvere}
\tan (3x+\ang{50})&=\tan(x-\ang{30})
\intertext{Discutiamo il lato sinistro}
3x+\ang{50}&=\ang{90}+k\ang{180}\\
3x&=\ang{90}-\ang{50}+k\ang{180}\\
x&=\dfrac{\ang{40}}{3}+k\ang{60}
\intertext{quindi}
x&\neq\dfrac{\ang{40}}{3}+k\ang{60}
\intertext{Discutiamo il lato destro}
x-\ang{30}&=\ang{90}+k\ang{180}\\
x&=\ang{90}+\ang{30}+k\ang{180}\\
x&=\ang{120}+k\ang{180}
\intertext{quindi}
x&\neq\ang{120}+k\ang{180}
\intertext{Uguagliamo gli argomenti e trovo le soluzioni}
3x+\ang{50}&=x-\ang{30}+k\ang{180}\\
3x-x&=-\ang{50}-\ang{30}+k\ang{180}\\
2x&=-\ang{80}+k\ang{180}
\intertext{La soluzione è}
x&=-\ang{40}+k\ang{90}
\intertext{La soluzione è accettabile perchè}
x&\neq\dfrac{\ang{40}}{3}+k\ang{180}\\
x&\neq\ang{120}+k\ang{180}
\end{align*}
\begin{table}
\[
\begin{array}{cccc}
\toprule
Equazione &Esistenza & Soluzione & Trasformazione \\ 
\midrule
\tan\alpha=\tan\beta &\alpha\neq\ang{90}+k\ang{180} &\alpha=\beta+k\ang{180} &\\
& \beta\neq\ang{90}+k\ang{180}& &\\
%&& &\\
\midrule
\tan\alpha=-\tan\beta & \alpha\neq\ang{90}+k\ang{180} &\alpha=\ang{180}-\beta+k\ang{180} &-\tan(\beta)=\tan(\ang{180}-\beta)\\
& \beta\neq\ang{90}+k\ang{180}& &\\
%&& &\\
\bottomrule
\end{array}
\] 
\caption{Equazioni elementari in tangente}\label{tab:EquazioniElementariInTangente}
\end{table}
\begin{esempiot}{Equazione goniometrica}{}
	Risolvere l'equazione $\tan (5x-\ang{210})=\sqrt{3}$
\end{esempiot}
\begin{align*}
\tan (5x-\ang{210})=&\sqrt{3}
\intertext{Discutiamo il lato sinistro}
5x-\ang{210}=&\ang{90}+k\ang{180}\\
5x=&\ang{90}+\ang{210}+k\ang{180}\\
x=&\ang{60}+k\ang{36}\\
x\neq&\ang{60}+k\ang{36}\\
\intertext{Risolviamo la soluzione}
\tan y=&\sqrt{3}\\
y=&\ang{60}+k\ang{180}\\
5x-\ang{210}=&\ang{60}+k\ang{180}\\
5x=&\ang{210}+\ang{60}+k\ang{180}\\
5x=&\ang{270}+k\ang{180}\\
x=&\ang{54}+k\ang{36}\\
\end{align*}
%%% Local Variables:
%%% mode: latex
%%% TeX-master: t
%%% End:
\begin{table}
	\centering
	\begin{tabular}{CLLL}
	\toprule
		 \multicolumn{1}{c}{Tipo} & \multicolumn{1}{c}{Equazione} & \multicolumn{2}{c}{Soluzioni} \\ 
		\midrule
		1&\sin\alpha=\sin\beta  &\alpha=\beta+k\ang{360}  & \alpha=\ang{180}-\beta+k\ang{360} \\ 
		2&\cos\alpha=\cos\beta  &\alpha=\beta+k\ang{360}  & \alpha=-\beta+k\ang{360} \\ 
		3&\tan\alpha=\tan\beta&\alpha=\beta+k\ang{180}  &\\
	4&\cot\alpha=\cot\beta&\alpha=\beta+k\ang{180}  &\\
\bottomrule
	\end{tabular} 
	\caption{Equazioni goniometriche elementari}\label{tab:Equazionigoniometricheelementari}
\end{table}
\begin{table}
	\centering
	\begin{tabular}{LLC}
		\toprule
		\multicolumn{1}{c}{Equazione} & \multicolumn{1}{c}{Trasformazione} & \multicolumn{1}{c}{Tipo}\\
			\midrule
		\sin\alpha=-\sin\beta &\sin\alpha=\sin(-\beta)&1\\ 
		\cos\alpha=-\cos\beta &\cos\alpha=\cos(\ang{180}-\beta)&2\\
		\tan\alpha=-\tan\beta &\tan\alpha=\tan(-\beta)&3\\
		\cot\alpha=-\cot\beta &\cot\alpha=\cot(-\beta)&4\\  
		\sin\alpha=\cos\beta &\sin\alpha=\sin(\ang{90}-\beta)&1\\
	  	\sin\alpha=\cos\beta &\cos(\ang{90}-\alpha)=\cos\beta&2\\
	  	\sin\alpha=-\cos\beta &-\sin(-\alpha)=-\cos\beta\quad\sin(-\alpha)=\sin(\ang{90}-\beta)&1\\
	
	
		\bottomrule
	\end{tabular} 
	\caption{Trasformazione equazioni goniometriche}\label{tab:Trasformazioneequazionigoniometriche}
\end{table}