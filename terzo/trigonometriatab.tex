\chapter{Trigonometria}
\label{cha:trigonometria}
\begin{figure}
	\centering
	\includestandalone{terzo/trigonometria/triangolopitagorico1}
	\caption{Triangolo rettangolo}
	\label{fig:triangolopitagorico1}
\end{figure}
\section{I triangoli rettangoli}
\begin{table}
\centering
\begin{tabular}{CCC}
\toprule
\multicolumn{3}{c}{TRIANGOLI RETTANGOLI} \tabularnewline
\midrule\tabularnewline
\multicolumn{1}{C}{\beta} &  & \multicolumn{1}{C}{\gamma} \tabularnewline\tabularnewline
b=a\sin\beta &  & c=a\sin\gamma \tabularnewline\tabularnewline
c=a\cos\beta &  & b=a\cos\gamma \tabularnewline\tabularnewline
b=c\tan\beta &  & c=b\tan\gamma \tabularnewline\tabularnewline
A=\dfrac{1}{2}ac\sin\beta &  & A=\dfrac{1}{2}ab\sin\gamma \tabularnewline\tabularnewline
\multicolumn{3}{C}{\beta+\gamma=\ang{90}} \tabularnewline\tabularnewline 
\multicolumn{3}{C}{2P=a+b+c} \tabularnewline\tabularnewline
%&&\\
\midrule
\multicolumn{3}{c}{TRIANGOLI QUALUNQUE} \tabularnewline
\midrule\tabularnewline
\multicolumn{1}{C}{\alpha} & \multicolumn{1}{C}{\beta} & \multicolumn{1}{C}{\gamma} \tabularnewline\tabularnewline
a^2=b^2+c^2-2bc\cos\alpha & b^2=a^2+c^2-2ac\cos\beta & c^2=a^2+b^2-2ab\cos\gamma \tabularnewline\tabularnewline
\multicolumn{3}{C}{\dfrac{a}{\sin\alpha}=\dfrac{b}{\sin\beta}=\dfrac{c}{\sin\gamma}} \tabularnewline\tabularnewline
A=\dfrac{1}{2}bc\sin\alpha & A=\dfrac{1}{2}ac\sin\beta & A=\dfrac{1}{2}ab\sin\gamma \tabularnewline\tabularnewline
\multicolumn{3}{C}{\alpha+\beta+\gamma=\ang{180}} \tabularnewline\tabularnewline
\multicolumn{3}{C}{2P=a+b+c} \tabularnewline\tabularnewline
\bottomrule
\end{tabular}
\caption{I triangoli}
\end{table}

Iniziamo con un po di notazione. I punti si indicano con le lettere maiuscole, la lunghezza dei segmenti con le  minuscole.  L'ampiezza degli angoli con le lettere greche. All'angolo di maggiore  ampiezza corrisponde la lettera $A$. Per rimanenti partendo da $A$ e muovendosi in senso antiorario, si assegnano gli altri vertici. Al vertice $A$ corrisponde l'angolo $\alpha$. Opposto ad $A$ vi è il lato $a$. Un triangolo rettangolo è formato da due lati chiamati cateti\index{Triangolo!rettangolo!cateto} e un lato chiamato ipotenusa. L'ipotenusa\index{Triangolo!rettangolo!ipotenusa} è il lato di lunghezza maggiore. 

I lati di un triangolo sono classificati, rispetto ad un angolo, come opposti o adiacenti. Guardando la figura\nobs\vref{fig:triangolooppostoadiacente}, vediamo che rispetto all'angolo $\gamma$ il cateto\index{Cateto!opposto}\index{Cateto!adiacente} $AC$ è adiacente perché lato dell'angolo, mentre $AB$ è opposto\index{Angolo!opposto}\index{Angolo!adiacente}  non facendo parte dell'angolo. 
 
 La figura\nobs\vref{fig:triangolopitagorico1} mostra come devono essere assegnati i nomi.
 
 La somma degli angoli interni\index{Angoli!interni!somma} di un triangolo è un angolo piatto. Quindi $\alpha+\beta+\gamma=\ang{180}$. 
\begin{figure}
	\centering
	\includestandalone{terzo/trigonometria/triangolooppostoadiacente}
	\caption{Elementi di un triangolo rettangolo}
	\label{fig:triangolooppostoadiacente}
\end{figure}
\subsection{Relazioni fondamentali}
\begin{teoremat}{}{}
Consideriamo la figura\nobs\vref{fig:triangolopitagorico1}. Valgono le seguenti relazioni fra i cateti gli angoli e l'ipotenusa
\begin{align*}
c&=a\sin\gamma&b&=a\cos\gamma\\
b&=a\sin\beta&c&=a\cos\beta
\end{align*}
\end{teoremat}
Quindi

Un cateto è uguale  al prodotto dell'ipotenusa per il seno dell'angolo opposto

\noindent oppure

Un cateto è uguale al prodotto dell'ipotenusa per il coseno dell'angolo opposto.

\noindent Valgono le seguenti relazioni
\begin{align*}
\dfrac{c}{a}&=\sin\gamma&\dfrac{b}{a}&=\cos\gamma\\
\dfrac{b}{a}&=\sin\beta&\dfrac{c}{a}&=\cos\beta
\end{align*}
Quindi

Il rapporto fra  cateto e l'ipotenusa è uguale al seno dell'angolo opposto

\noindent oppure

Il rapporto fra  cateto e l'ipotenusa è uguale al coseno dell'angolo adiacente.

\noindent Dalle relazioni di partenza si ottiene
\[\dfrac{b}{sen\beta}=\dfrac{b}{\cos\gamma}=\dfrac{c}{\sin\gamma}=\dfrac{c}{cos\beta}=a \]
\begin{teoremat}{}{}
Dividendo fra loro le relazioni di partenza otteniamo
\begin{align*}
b&=c\tan\beta&c&=b\tan\gamma\\
b&=c\cot\gamma&c&=b\cot\beta
\end{align*}
\end{teoremat}
\section{Risoluzione triangoli rettangoli}
La risoluzione di un triangolo consiste nel trovare tutti i suoi elementi conoscendone alcuni. Ricordiamo che in un triangolo rettangolo vale il teorema di Pitagora.\index{Teorema!Pitagora} Quindi, facendo riferimento alla figura\nobs\vref{fig:TeoremaPitagora_1}:
\begin{align*}
a^2&=b^2+c^2\\
a{}&=\sqrt{b^2+c^2}\\
b^2&=a^2-c^2\\
b{}&=\sqrt{a^2-c^2}\\
c^2&=a^2-b^2\\
c{}&=\sqrt{a^2-b^2}\\
\end{align*}
\begin{figure}
	\centering
	\includestandalone{terzo/trigonometria/pitagora_1}
	\caption{Teorema di Pitagora}
	\label{fig:TeoremaPitagora_1}
\end{figure}
Inoltre dato che la somma degli angoli interni di un triangolo è 
$\ang{180}$ la somma dei due angoli acuti è di $\ang{90}$. quindi
\[\beta+\gamma=\ang{90}\]
\subsection{Angolo acuto e ipotenusa noti}
Risolviamo questo caso, conosciamo, come nella figura\nobs\vref{fig:risTriangRett_1}, l'ipotenusa $a$ e un angolo acuto, per esempio\nobs$\gamma$.
\begin{align*}
\beta&=\ang{90}-\gamma&\beta&=\ang{90}-\gamma\\
c&=a\sin\gamma&c&=a\cos\beta\\
b&=a\cos\gamma&b&=a\cos\beta
\end{align*}
\begin{esempiot}{Risolvere triangolo rettangolo}{}
Trovare gli altri elementi di un triangolo sapendo che l'ipotenusa $a=3$ e l'angolo acuto $\gamma=\ang{30}$ 
\end{esempiot}
\begin{align*}
a&=3\\
\gamma&=\ang{30}\\
\beta&=\ang{90}-\ang{30}=\ang{60}\\
c&=a\sin\gamma\\
c&=3\sin\ang{30}=3\cdot\dfrac{1}{2}=\dfrac{3}{2}\\
b&=a\cos\gamma\\
b&=3\cos\ang{30}=3\cdot\dfrac{\sqrt{3}}{2}=\dfrac{3\sqrt{3}}{2}
\end{align*}
\subsection{Angolo acuto e cateto noti}
Risolviamo questo caso, conosciamo, come nella figura\nobs\vref{fig:risTriangRett_2}, un cateto $b$ e un angolo acuto, ad esempio $\gamma$.
\begin{align*}
\beta&=\ang{90}-\gamma&\beta&=\ang{90}-\gamma\\
c&=b\tan\gamma&c&=a\cot\beta\\
a&=\dfrac{b}{\sin\beta}&a&=\dfrac{b}{\cos\beta}
\end{align*}
\begin{esempiot}{Risolvere triangolo rettangolo}{}
Trovare gli elementi ignoti di un triangolo rettangolo, sapendo che  il cateto $b=5$ e l'angolo $\gamma=\ang{60}$.
\end{esempiot}
\begin{align*}
b&=5\\
\gamma &=\ang{60}\\
\beta&=\ang{90}-\ang{60}=\ang{30}\\
c&=b\tan\gamma\\
c&=5\tan\ang{60}=5\cdot\sqrt{5}=5\sqrt{3}\\
a&=\dfrac{b}{\sin\beta}\\
a&=\dfrac{5}{\sin\ang{30}}=\dfrac{5}{\dfrac{1}{2}}=10
\end{align*}
Una variante di quanto prima è quello che segue. Qui un angolo è noto tramite il valore di una funzione goniometrica.
\begin{esempiot}{Trovare gli elementi ignoti di un triangolo rettangolo}{}
Trovare gli elementi ignoti di un triangolo rettangolo, sapendo che  il cateto $b=4$ e l'angolo $\cos\gamma=\dfrac{4}{5}$.
\end{esempiot}
\begin{align*}
\cos\gamma&=\dfrac{4}{5}\\
b&=4
\intertext{determiniamo l'ipotenusa $a$}
b&=a\cos\gamma\\
4&=a\dfrac{4}{5}\\
\intertext{l'ipotenusa è:}
a&=\dfrac{20}{4}=5\\
\intertext{Per determinare il cateto $c$ ho bisogno di $\sin\gamma$}
\sin\gamma&=\sqrt{1-\cos^2\gamma}\\
&=\sqrt{1-\dfrac{16}{25}}\\
&=\sqrt{\dfrac{25-16}{25}}\\
&=\sqrt{\dfrac{9}{25}}=\dfrac{3}{5}\\
\intertext{ora posso trovare il cateto $c$}
c&=a\sin\gamma\\
c&=5\dfrac{3}{5}=3
\end{align*}
\subsection{Ipotenusa e cateto}
Risolviamo questo caso, conosciamo, come nella figura\nobs\vref{fig:risTriangRett_3}, un cateto $b$ e l'ipotenusa $a$.
\begin{align*}
\beta&=\arcsin\dfrac{b}{a}&\gamma&=\arccos\dfrac{b}{a}\\
\gamma&=\ang{90}-\beta&\beta&=\ang{90}-\gamma\\
c&=a\cos\beta&c&a\sin\gamma
\end{align*}
\begin{esempiot}{Risolvere triangolo rettangolo 1}{}
Trovare gli elementi ignoti di un triangolo rettangolo sapendo che l'ipotenusa $a=2$  il cateto $b=\sqrt{5}$.
\end{esempiot}
\begin{align*}
a&=\num{2}\\
b&=\sqrt{\num{3}}\\
\beta&=\arcsin\dfrac{b}{a}\\
\beta&=\arcsin\dfrac{\sqrt{3}}{2}=\ang{60}\\
\gamma&=\ang{90}-\ang{60}=\ang{30}\\
c&=a\cos\beta\\
c&=2\cos\ang{60}=2\cdot\dfrac{1}{2}=1
\end{align*} 
\subsection{Cateti noti}
Risolviamo questo caso, conosciamo, come nella figura\nobs\vref{fig:risTriangRett_4}, con i cateti $b$ e $c$ noti.
\begin{align*}
\gamma&=\arctan\dfrac{c}{b}&\beta&=\arctan\dfrac{b}{c}\\
\beta&=\ang{90}-\gamma&\gamma&=\ang{90}-\beta\\
a&=\dfrac{c}{\sin\gamma}&a&=\dfrac{c}{\cos\beta}
\end{align*}
\begin{esempiot}{Risolvere triangolo rettangolo 2}{}
Trovare gli elementi ignoti di un triangolo rettangolo sapendo che  il cateto $b=5$ e il lato $c=12$
\end{esempiot}
\begin{align*}
b&=5\\
c&=12\\
\gamma&=\arctan\dfrac{c}{b}\\
\gamma&=\arctan\dfrac{12}{5}\approxeq\ang{67.38}\\
\beta&=\ang{90}-\gamma=\ang{67.38}\\
a&=\dfrac{c}{\sin\gamma}\\
a&=\dfrac{12}{\sin\ang{67.38}}\approxeq\dfrac{12}{\num{0.92}}\approxeq\num{13.0}
\end{align*}
\begin{figure}
	\begin{subfigure}[b]{.5\linewidth}
		\centering
\includestandalone{terzo/trigonometria/risTriangRett_1}
	\caption{Ipotenusa e angolo acuto noto}
	\label{fig:risTriangRett_1}
	\end{subfigure}%
	\begin{subfigure}[b]{.5\linewidth}
		\centering
		\includestandalone{terzo/trigonometria/risTriangRett_2}
		\caption{Cateto e angolo acuto noto}
		\label{fig:risTriangRett_2}
	\end{subfigure}
	\begin{subfigure}[b]{.5\linewidth}
		\centering
	\includestandalone{terzo/trigonometria/risTriangRett_3}
	\caption{Ipotenusa e cateto noto}
	\label{fig:risTriangRett_3}
	\end{subfigure}%
	\begin{subfigure}[b]{.5\linewidth}
		\centering
		\includestandalone{terzo/trigonometria/risTriangRett_4}
		\caption{Cateti noti}
		\label{fig:risTriangRett_4}
	\end{subfigure}
	\captionof{figure}{Risoluzione triangoli rettangoli}
	\label{fig:RisoluzioneTriangoliRettangoli}
\end{figure}
\section{Triangoli qualunque}
\subsection{Teorema dei seni}
\begin{figure}
	\centering
	\includestandalone{terzo/trigonometria/TeoSeni_1}
	\caption{Teorema dei seni}
	\label{fig:TeoremDeiSeni}
\end{figure}
Per un triangolo qualunque\index{Teorema!seni} valgono le seguenti uguaglianze\[\dfrac{a}{\sin\alpha}=\dfrac{b}{\sin\beta}=\dfrac{c}{\sin\beta}=2R \]

Quindi in un triangolo qualunque il rapporto fra il lato e il seno dell'angolo opposto è costante e uguale al diametro della circonferenza circoscritta. La figura\nobs\vref{fig:TeoremDeiSeni} mostra le relazioni.

Possiamo riscrivere il teorema dei seni in questa maniera: 
\begin{align*}
a&=b\dfrac{\sin\alpha}{\sin\beta}& a&=c\dfrac{\sin\alpha}{\sin\gamma}\\
b&=a\dfrac{\sin\beta}{\sin\beta}& b&=c\dfrac{\sin\beta}{\sin\gamma}\\
c&=a\dfrac{\sin\gamma}{\sin\alpha}& c&=b\dfrac{\sin\gamma}{\sin\beta}
\end{align*}
o in questa
\begin{align*}
\sin\alpha&=\dfrac{a}{b}\sin\beta&\sin\alpha&=\dfrac{a}{c}\sin\gamma\\
\sin\beta&=\dfrac{b}{a}\sin\alpha&\sin\beta&=\dfrac{b}{c}\sin\gamma\\
\sin\gamma&=\dfrac{c}{a}\sin\alpha&\sin\gamma&=\dfrac{c}{b}\sin\beta\\
\end{align*}
\subsection{Teorema di Carnot}
\begin{figure}
	\centering
	\includestandalone{terzo/trigonometria/canot_1}
	\caption{Teorema di Carnot}
	\label{fig:TeoremDiCarnot_1}
\end{figure}
Una versione più generale del teorema di Pitagora è il teorema di Carnot\index{Teorema!Carnot}. Partendo dalla figura\nobs\vref{fig:TeoremDiCarnot_1} avremo queste relazioni.
\begin{align*}
a^2&=b^2+c^2-2bc\cos\alpha\\
b^2&=a^2+c^2-2ac\cos\beta\\
c^2&=a^2+b^2-2ab\cos\gamma
\end{align*}
Possiamo scrivere le precedenti equazioni isolando i coseni cioè:
\begin{align*}
\cos\alpha&=\dfrac{b^2+c^2-a^2}{2bc}\\
\cos\beta&=\dfrac{a^2+c^2-b^2}{2ac}\\
\cos\gamma&=\dfrac{a^2+b^2-c^2}{2ab}\\
\end{align*}
queste soluzioni possono servire per definire gli angoli del triangolo cioè:
\begin{align*}
\alpha&=\arccos(\dfrac{b^2+c^2-a^2}{2bc})\\
\beta&=\arccos(\dfrac{a^2+c^2-b^2}{2ac})\\
\gamma&=\arccos(\dfrac{a^2+b^2-c^2}{2ab})\\
\end{align*}
\begin{figure}
	\begin{subfigure}[b]{.5\linewidth}
		\centering
\includestandalone{terzo/trigonometria/risTriangQualunque_1}
	\caption{Un lato e due angoli noti}
	\label{fig:risTriangQqualunque_1}
	\end{subfigure}%
	\begin{subfigure}[b]{.5\linewidth}
		\centering
	\includestandalone{terzo/trigonometria/risTriangQualunque_2}
		\caption{Due lati l'angolo fra loro compreso noti}
		\label{fig:risTriangQqualunque_2}
	\end{subfigure}
	\begin{subfigure}[b]{.5\linewidth}
		\centering
		\includestandalone{terzo/trigonometria/risTriangQualunque_3}
		\caption{Due lati l'angolo fra loro non compreso noti}
		\label{fig:risTriangQqualunque_3}
	\end{subfigure}%
	\begin{subfigure}[b]{.5\linewidth}
		\centering
		\includestandalone{terzo/trigonometria/risTriangQualunque_4}
		\caption{Due lati l'angolo fra loro non compreso noti}
		\label{fig:risTriangQqualunque_4}
	\end{subfigure}
	\captionof{figure}{Risoluzione triangoli qualunque}
	\label{fig:RisoluzioneTriangoliQualunque}
\end{figure}
\section{Risoluzione di triangolo qualunque}
\subsection{Un lato e due angoli}
Se è noto il lato $c$ e gli angoli $\alpha$ e $\beta$ come nella figura\nobs\vref{fig:risTriangQqualunque_1} avremo:
\begin{align*}
\gamma&=\ang{180}-(\alpha+\beta)\\
a&=c\dfrac{\sin\alpha}{\sin\gamma}\\
b&=c\dfrac{\sin\beta}{\sin\gamma}\\
\end{align*}
\subsection{Due lati e l'angolo fra essi compreso} 
In questo caso supponiamo noti i lati $b$ e $c$ e l'angolo $\alpha$ fra loro compreso, come nella figura\nobs\vref{fig:risTriangQqualunque_2} avremo:
\begin{align*}
a&=\sqrt{b^2+c^2-2bc\cos\alpha}\\
\beta&=\arccos(\dfrac{a^2+c^2-b^2}{2ac})\\
\gamma&=\arccos(\dfrac{a^2+b^2-c^2}{2ab})
\intertext{verificando che}
\alpha+&\beta+\gamma=\ang{180}
\end{align*}
\subsection{Due lati e l'angolo apposto a quello compreso}
In questo caso supponiamo noti i lati $b$ e $c$ e l'angolo $\beta$ fra loro non compreso, come nella figura\nobs\vref{fig:risTriangQqualunque_3} avremo:
\begin{align*}
\intertext{verificando le soluzioni}
\gamma&=\arcsin(\dfrac{c}{b}{\sin\beta}) \\
\alpha&=\ang{180}-(\beta+\gamma)\\
a&=b\dfrac{\sin\alpha}{\sin\beta}
\end{align*}
\subsection{Tre lati}
In questo caso supponiamo noti i lati $a$ $b$ e $c$  come nella figura\nobs\vref{fig:risTriangQqualunque_4} avremo:
\begin{align*}
\alpha&=\arccos(\dfrac{b^2+c^2-a^2}{2bc})\\
\beta&=\arccos(\dfrac{a^2+c^2-b^2}{2ac})\\
\gamma&=\arccos(\dfrac{a^2+b^2-c^2}{2ab})\\
\end{align*}
\chapter{Forma goniometrica dei numeri complessi}
\begin{definizionet}{Forma goniometrica}{}
Un numero complesso $z\in\Co$ è in forma goniometrica se
\[z=r\left[\cos\theta+\uimm\sin\theta \right] \] dove
\begin{itemize}
\item $\uimm$ unità immaginaria
\item $r\in\R\; r\geq 0$ detto modulo
\item $-\pi <\theta\leq\pi\; \vee\; 0\leq\theta<2\pi$ detto anomalia.
\end{itemize}
\end{definizionet}