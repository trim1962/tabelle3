\chapter{Angoli}
\label{cha:angoli}
\section{Definizioni}
\begin{definizionet}{Angolo}{}
	Un angolo\index{Angolo} è la parte di piano compreso fra due semirette dette lati. I lati hanno in comune l'origine chiamata vertice.
\end{definizionet}

Due semirette formano due angoli: uno convesso,\index{Angolo!convesso} l'altro concavo\index{Angolo!concavo} come nella\nobs\vref{fig:angconconvposoneg}. 
\begin{figure} %[htbp]
	\centering
	\includestandalone[width=.6\textwidth]{funzgonioTikz/angoliconcaviconvessi}
	\caption{Angoli concavi e convessi, positivi e negativi}\label{fig:angconconvposoneg}
\end{figure}
\begin{figure} %[htbp]
	\centering
	\includestandalone[width=0.9\textwidth]{funzgonioTikz/angolinotevoli}
	\caption{Angoli notevoli}\label{fig:Angolorettoposneggonio}
\end{figure}
\begin{figure}
	\includestandalone[width=\textwidth]{funzgonioTikz/mappe_concettuali_Angoli_1}
	\caption{Mappa goniometria l'angolo}\label{fig:MappaGonometria1}
\end{figure}
\begin{definizionet}{Angoli positivi e negativi}{}
	Fissato un lato, l'angolo è positivo\index{Angolo!positivo} se per costruirlo ruotiamo l'altra semiretta in senso antiorario. Un angolo è negativo\index{Angolo!negativo} se ruoteremo l'altro lato in senso orario come nella\nobs\vref{fig:angconconvposoneg}. 
\end{definizionet}
\begin{definizionet}{Angolo giro}{}
	Un angolo è giro\index{Angolo!giro} quando dopo la rotazione, le due semirette coincidono. 
\end{definizionet}
\begin{definizionet}{Angolo piatto}{}
	Un angolo è piatto\index{Angolo!retto} quando i suoi lati sono sulla stessa retta.
\end{definizionet}
\begin{definizionet}{Angolo retto}{}
	Un angolo è retto\index{Angolo!retto} quando è la metà di un angolo piatto. 
\end{definizionet}
\begin{definizionet}{Angolo acuto}{}
	Un angolo è acuto\index{Angolo!acuto} se minore di un angolo retto.
\end{definizionet}
\begin{definizionet}{Angolo ottuso}{}
	Un angolo è ottuso\index{Angolo!ottuso} se maggiore di un angolo retto.
\end{definizionet}
La\nobs\vref{fig:Angolorettoposneggonio} mostra i vari casi.
\section{Misura dell'angolo}
\label{sec:MisuraAngoloGonio}
A ogni angolo viene associata una grandezza detta ampiezza. Nel SI, il radiante è l'unità di misura per l'angolo piatto.

Storicamente, sono accettate altre unità di misura come il grado sessagesimale, il grado sessadecimale  e infine il grado centesimale.\footnote{Il grado centesimale non verrà trattato. l'angolo centesimale è definito come la quattrocentesima parte di una angolo giro.}
\subsection{Angolo sessagesimale}
\begin{definizionet}{Grado sessagesimale}{}
	Un grado\index{Grado!sessagesimale} è la trecentosessantesima parte in un angolo giro. Il grado si suddivide in minuti e secondi. Il minuto è la sessantesima parte di un grado. Il secondo è la sessantesima parte di un minuto. Il secondo è suddiviso in decimi e centesimi.
\end{definizionet}
Quindi
\begin{align*}
\ang{;1;}=&\dfrac{\ang{1}}{60}\\
\ang{;;1}=&\dfrac{\ang{;1;}}{60}=\dfrac{\ang{1}}{3600}
\end{align*}
\begin{esempiot}{Angoli sessagesimali}{}\index{Grado!sessagesimale}
	L'angolo \ang{45;30;20} ha l'ampiezza di \ang{45} gradi \ang{;30;} minuti o primi e \ang{;;20} secondi. L'angolo \ang{30;45;23,7} secondo ha l'ampiezza di \ang{30} gradi \ang{;45;} primi e \ang{;;23} secondi e $7$ decimi.
\end{esempiot}
Un angolo giro\index{Angolo!giro} ha un'ampiezza di \ang{360} gradi. L'angolo piatto\index{Angolo!piatto}, metà di un angolo giro, ha quindi, l'ampiezza di \ang{180} gradi. L'angolo retto\index{Angolo!retto} metà di una angolo piatto, ha un'ampiezza di \ang{90} gradi.
\subsection{Angolo sessadecimale}
\begin{definizionet}{Grado sessadecimale}{}\index{Grado!sessadecimale}
	Un grado\index{Grado!sessadecimale} è la trecentosessantesima parte in un angolo giro. Non ha sottomultipli come il grado sessagesimale.
\end{definizionet}

\subsection{Radiante}
\begin{definizionet}{Radiante}{}\index{Radiante}
	Data una circonferenza di raggio $r$ e un angolo $\alpha$ con il vertice nel centro $C$ della circonferenza, come nella\nobs\vref{fig:radinatidefgonio}, se $l$ è la lunghezza dell'arco di circonferenza sotteso dall'angolo, chiamiamo radiante il rapporto \[\rho=\dfrac{l}{r} \]
\end{definizionet}
Dato che il radiante è il quoziente fra due lunghezze da ciò segue che è una grandezza adimensionale.
\begin{figure}
	\centering
	\includestandalone[width=0.6\textwidth]{funzgonioTikz/radianti}
	\caption{Radianti}
	\label{fig:radinatidefgonio}
\end{figure}
Avremo quindi che un angolo ha l'ampiezza di un radiante\index{Radiante} se l'arco di circonferenza $l$ è uguale al raggio $r$.
In un angolo giro\index{Angolo!giro} l'arco è lungo quanto la circonferenza. La sua misura in radianti è quindi:\[\rho=\dfrac{2\pi r}{r}=2\pi\]
Un angolo piatto, meta di un giro, misura \[\rho=\pi\]\index{Angolo!piatto} e un angolo
retto\index{Angolo!retto} misura: \[\rho=\dfrac{\pi}{2} \] 
\section{Conversioni}
\subsection{Angoli sessagesimali e sottomultipli}
La conversione fra i gradi e i suoi sottomultipli è ben illustrata dalla~\cref{fig:Convertiregradiprimisecondi}. 
Quindi quando si scende si moltiplica il valore dato per \num{60} mentre se si sale si divide la misura per \num{60}.
\begin{figure} %[htbp]
	\centering
	\includestandalone[width=0.9\textwidth]{funzgonioTikz/gradiprimisecondiconversione}
	\caption{Convertire gradi primi e secondi}\label{fig:Convertiregradiprimisecondi}
\end{figure}
\begin{esempiot}{Conversione gradi, minuti e secondi}{}
Se un angolo ha un'ampiezza di \ang{134} a quanti minuti corrisponde e a quanti secondi?
\end{esempiot}
Semplicemente 
\begin{align*}
	\intertext{i minuti}
	\ang{134}\times 60=&\ang{;8040;}\\
		\intertext{i secondi}
	\ang{;8040;}\times  60=&\ang{;;482400}\\
\end{align*}
usando la calcolatrice
\begin{center}
	\begin{tabular}{ll}
	\tasto{134}\tastoper\tasto{60}\tastouguale & \num{8040} \\ 
	\tastoans\tastoper\tasto{60}\tastouguale & \num{482400} \\
\end{tabular}
\end{center}
\begin{esempiot}{Conversione secondi, minuti e gradi}{}
	Se un angolo ha un'ampiezza di \ang{;;11292} a quanti minuti corrisponde e a quanti gradi?
\end{esempiot}
\begin{align*}
\intertext{Da secondi a minuti}
\ang{;;11292}\div 60=&\num{188.2}
\intertext{Quindi il \num{60} sta \num{188} volte nel \num{11292} }\\
\intertext{quindi il resto è}
\num{11292}-\num{60}\times\num{188}=&12\\
\intertext{quindi}\\
\ang{;;11292}=&\ang{;188;12}
\end{align*}
\begin{align*}
\intertext{Da minuti a gradi}
\ang{;188;}\div 60=&3.1\overline{3}
\intertext{Quindi il \num{60} sta \num{3} volte nel \num{188} }
\intertext{quindi il resto è}
\num{188}-\num{60}\times\num{3}=&8\\
\intertext{quindi}
\ang{;188;}=&\ang{3;8;}
\end{align*}
Ricapitolando:
\[\ang{;;11292}=\ang{3;8;12}\]
\subsection{Da sessagesimale a sessa-decimale}
\begin{esempiot}{Conversione}{}
Convertire in forma decimale un angolo di ampiezza pari a \ang{44;58;48} arrotondato alla \tlungarrotandamento
\end{esempiot}
Abbiamo un angolo di ampiezza pari a \ang{44;58;48} e vogliamo
scriverlo in forma decimale\index{Grado!forma decimale}. Dato che $\ang{;1;}=\dfrac{\ang{1}}{60}$ e che $\ang{;;1}=\dfrac{\ang{;1;}}{60}=\dfrac{\ang{1}}{3600}$ avremo
\begin{align*}
\alpha=&\ang{44}+\left(\dfrac{58}{60}\right)^{\si{\degree} }+\left(\dfrac{48}{3600}\right)^{\si{\degree} }\\
=&\ang{44}+\left(\dfrac{58\cdot 60+48}{3600}\right)^{\si{\degree}}\\
=&\ang{44}+\left(\dfrac{3528}{3600}\right)^{\si{\degree}}\approx\ang[round-precision=\lungarrotandamento,round-mode=places]{44,98}
\end{align*}
usando la calcolatrice
\begin{center}
\begin{tabular}{ll}
	\tasto{58}\tastoper\tasto{60}\tastouguale & 3480 \\ 
	\tastoans\tastopiu\tasto{48}\tastouguale & 3528 \\
	\tastoans\tastodiv\tasto{3600}\tastouguale & \num[round-precision=\lungarrotandamento,round-mode=places]{0.98} \\
	\tastoans\tastopiu\tasto{44}\tastouguale&\num[round-precision=\lungarrotandamento,round-mode=places]{44.98} \\
\end{tabular}
\end{center} 
\begin{esempiot}{Convertire in forma sessagesimale}{}
Convertiamo $7,42^{\circ}$ in gradi minuti e secondi:
\end{esempiot}
\begin{align*}
\alpha^{\si{\degree}}&=\ang{7,42}-\ang{7}=\ang{0.42}\\ 
=&\ang{0.42}\cdot 60=\ang{;25.2;}\\
=&\ang{;25.2;}-\ang{;25;}=\ang{;0.2;}\\
=&\ang{;0.2;}\cdot 60=\ang{;;12}\\
\end{align*}
Quindi \[\alpha=\ang{7;25;12}\]
usando la calcolatrice
\begin{center}
\begin{tabular}{ll}
	\tasto{7.42}\tastomeno\tasto{7}\tastouguale & \num{0.42} \\ 
	\tastoans\tastoper\tasto{60}\tastouguale & \num{25.2} \\
	\tasto{25.2}\tastomeno\tasto{25}\tastouguale & \num{0.2} \\ 
	\tastoans\tastoper\tasto{60}\tastouguale & \num{12} \\
\end{tabular}
\end{center} 
\subsection{Gradi radianti }
Per convertire da gradi sessagesimali a radianti\index{Grado!radiante!conversione} si procede in questo modo:
\begin{align*}
\dfrac{l}{2\pi r}&=\dfrac{\alpha}{\ang{360}}\\
\dfrac{\rho}{2\pi}&=\dfrac{\alpha}{\ang{360}}\\
\rho&=\dfrac{\alpha 2\pi}{\ang{360}}\\
\rho&=\dfrac{\pi}{\ang{180}}\alpha
\intertext{segue che per passare da radianti a gradi sessagesimali avremo}
\alpha&=\dfrac{\ang{180}}{\pi}\rho
\end{align*}
Alcuni semplici esempi di conversione fra angoli e radianti
\begin{esempiot}{Radianti in gradi}{}
	Quanto corrisponde in gradi un radiante?
\end{esempiot} 
\begin{align*}
\alpha=&\dfrac{180}{\pi}\cdot 1^{\si{\degree} }\\
\approx&\ang[round-precision=\lungarrotandamento,round-mode=places]{57.29577951}\cdot\ang{1}\\
\approx&\ang[round-precision=\lungarrotandamento,round-mode=places]{57.29577951}\\
\end{align*}
\begin{center}
	\begin{tabular}{ll}
		\tasto{180}\tastodiv\tastopgreco\tastoper\tasto{1}\tastouguale&
		\tasto{\num[round-precision=\lungarrotandamento,round-mode=places]{57.29577951}}\\
	\end{tabular} 
\end{center}
\begin{esempiot}{Gradi in radianti}{}
	Quanto corrisponde in radianti un grado?
\end{esempiot}
\begin{align*}
\rho=&\dfrac{\pi}{180}\cdot 1\\
\approx&\ang[round-precision=\lungarrotandamento,round-mode=places]{0.017453292}\cdot 1\\
\approx&\ang[round-precision=\lungarrotandamento,round-mode=places]{0.017453292}\\
\end{align*}
\begin{center}
	\begin{tabular}{ll}
		\tastopgreco\tastodiv\tasto{180}\tastoper\tasto{1}\tastouguale&
		\tasto{\num[round-precision=\lungarrotandamento,round-mode=places]{0.017453292}}\\
	\end{tabular} 
\end{center}
\section{Operazioni con i gradi sessagesimali}
\subsection{Somma}
La somma fra due angoli sessagesimali equivalente a tre somme: secondi con secondi, minuti con i minuti, i gradi con i gradi. Se nel sommare otteniamo un valore maggiore a $60$ bisogna aggiungere un'unità di ordine superiore.
\begin{esempiot}{Somma di angoli}{}
	Trovare la somma $\alpha=\ang{30;28;15}$
\end{esempiot}