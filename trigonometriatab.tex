\chapter{Trigonometria}
\label{cha:trigonometria}
\begin{figure}
	\centering
	\includestandalone{trigonometria/triangolopitagorico1}
	\caption{Triangolo rettangolo}
	\label{fig:triangolopitagorico1}
\end{figure}
\begin{table}
	\centering
	\begin{tabular}{CCC}
		\toprule
		\multicolumn{3}{c}{TRIANGOLI RETTANGOLI} \tabularnewline
		\midrule\tabularnewline
		\multicolumn{1}{C}{\beta} &  & \multicolumn{1}{C}{\gamma} \tabularnewline\tabularnewline
		b=a\sin\beta &  & c=a\sin\gamma \tabularnewline\tabularnewline
		c=a\cos\beta &  & b=a\cos\gamma \tabularnewline\tabularnewline
		b=c\tan\beta &  & c=b\tan\gamma \tabularnewline\tabularnewline
		A=\dfrac{1}{2}ac\sin\beta &  & A=\dfrac{1}{2}ab\sin\gamma \tabularnewline\tabularnewline
		\multicolumn{3}{C}{\beta+\gamma=\ang{90}} \tabularnewline\tabularnewline 
		\multicolumn{3}{C}{2P=a+b+c} \tabularnewline\tabularnewline
		%&&\\
		\midrule
		\multicolumn{3}{c}{TRIANGOLI QUALUNQUE} \tabularnewline
		\midrule\tabularnewline
		\multicolumn{1}{C}{\alpha} & \multicolumn{1}{C}{\beta} & \multicolumn{1}{C}{\gamma} \tabularnewline\tabularnewline
		a^2=b^2+c^2-2bc\cos\alpha & b^2=a^2+c^2-2ac\cos\beta & c^2=a^2+b^2-2ab\cos\gamma \tabularnewline\tabularnewline
		\multicolumn{3}{C}{\dfrac{a}{\sin\alpha}=\dfrac{b}{\sin\beta}=\dfrac{c}{\sin\gamma}} \tabularnewline\tabularnewline
		A=\dfrac{1}{2}bc\sin\alpha & A=\dfrac{1}{2}ac\sin\beta & A=\dfrac{1}{2}ab\sin\gamma \tabularnewline\tabularnewline
		\multicolumn{3}{C}{\alpha+\beta+\gamma=\ang{180}} \tabularnewline\tabularnewline
		\multicolumn{3}{C}{2P=a+b+c} \tabularnewline\tabularnewline
		\bottomrule
	\end{tabular}
	\caption{I triangoli}
\end{table}
\section{I triangoli}
Iniziamo con un po' di notazione. In un triangolo i  vertici si indicano con le lettere maiuscole, la lunghezza dei segmenti con le  minuscole, per  l'ampiezza degli angoli si usa l'alfabeto greco.  Allo spigolo $A$  corrisponde l'angolo di maggiore  ampiezza. Per rimanenti vertici, partendo  da $A$ e muovendosi in senso antiorario, si assegnano le lettere restanti. Al vertice $A$ corrisponde l'angolo di ampiezza $\alpha$, allo spigolo $B$ $\beta$ e a $C$ $\gamma$. Opposto al vertice $A$ vi è il lato $a$, gli altri lati a seguire.\par 
I lati di un triangolo sono classificati, rispetto a un angolo, come opposti o  adiacenti. Guardando la figura\nobs\vref{fig:triangolooppostoadiacente}, diciamo che rispetto all'angolo $\gamma$, il segmento $AC$ è adiacente perché appartiene a un  lato dell'angolo.\par  Il segmento  $AB$ è opposto a$\gamma$\index{Angolo!opposto}\index{Angolo!adiacente} perché  non appartiene ai lati  dell'angolo.\par
Un angolo è compreso fra due segmenti se questi appartengono ai suoi lati. L'angolo $\alpha$ è compreso tra i lati $AB$ e $AC$.\par 
 La figura\nobs\vref{fig:triangolopitagorico1} mostra come devono essere assegnati i nomi.
La somma degli angoli interni\index{Angoli!interni!somma} di un triangolo è un angolo piatto. Quindi \[\alpha+\beta+\gamma=\ang{180}\]. 
\begin{figure}
	\centering
	\includestandalone{trigonometria/triangolooppostoadiacente}
	\caption{Elementi di un triangolo rettangolo}
	\label{fig:triangolooppostoadiacente}
\end{figure}
\subsection{Area}
\begin{teoremat}{Area di un triangolo}{}
	L'area di un triangolo è uguale al semi prodotto di due lati per il seno dell'angolo fra essi compreso. 
	Quindi l'area del triangolo è:
	\begin{align*}
	A=&\dfrac{1}{2}bc\sin\alpha& A=&\dfrac{1}{2}ac\sin\beta& A=&\dfrac{1}{2}ab\sin\gamma \\
	\end{align*}
\end{teoremat}	
\section{Triangoli rettangoli}
 Un triangolo rettangolo è formato da due lati chiamati cateti\index{Triangolo!rettangolo!cateto} e da un lato, il maggiore, che è detto ipotenusa\index{Triangolo!rettangolo!ipotenusa}.\par Avremo:\begin{align*}
 \alpha=&{}\ang{90}&&&\alpha=&{}\frac{\pi}{2}\\
 \beta+\gamma=&{}\ang{90}&&&\beta+\gamma=&{}\frac{\pi}{2}\\
 &&A={}&\frac{bc}{2}\\
 &&a^2={}&b^2+c^2\\
  &&a={}&\sqrt{b^2+c^2}\\
  &&b={}&\sqrt{a^2-c^2}\\
  &&c={}&\sqrt{a^2-b^2}\\
  &&2P=&a+b+c
 \end{align*}
\subsection{Relazioni fondamentali}
\begin{teoremat}{Relazioni fondamentali}{}
Consideriamo la figura\nobs\vref{fig:triangolopitagorico1}. Valgono le seguenti relazioni fra i cateti gli angoli e l'ipotenusa
\begin{align*}
c=&a\sin\gamma&b=&a\cos\gamma\\
b=&a\sin\beta&c=&a\cos\beta
\end{align*}
\end{teoremat}
\noindent Quindi:\par
Un cateto è uguale  al prodotto dell'ipotenusa per il seno dell'angolo opposto.\par 
\noindent Oppure:\par 
Un cateto è uguale al prodotto dell'ipotenusa per il coseno dell'angolo adiacente.\par
\noindent Valgono le seguenti relazioni
\begin{align*}
\dfrac{c}{a}=&\sin\gamma&\dfrac{b}{a}=&\cos\gamma\\
\dfrac{b}{a}=&\sin\beta&\dfrac{c}{a}=&\cos\beta
\end{align*}
\noindent Quindi:\par
Il rapporto fra  cateto e l'ipotenusa è uguale al seno dell'angolo opposto.\par
\noindent Altrimenti:\par
Il rapporto fra  cateto e l'ipotenusa è uguale al coseno dell'angolo adiacente.\par 
\noindent Dalle relazioni di partenza si ottiene:
\[\dfrac{b}{sen\beta}=\dfrac{b}{\cos\gamma}=\dfrac{c}{\sin\gamma}=\dfrac{c}{cos\beta}=a \]
\begin{teoremat}{Relazioni derivate}{}
Dividendo fra loro le relazioni di partenza otteniamo
\begin{align*}
b=&c\tan\beta&c=&b\tan\gamma\\
b=&c\cot\gamma&c=&b\cot\beta
\end{align*}
\end{teoremat}
\noindent Quindi:\par
Un cateto è uguale all'altro cateto per la tangente dell'angolo opposto.\par \noindent Altrimenti:\par 
Un cateto è uguale all'altro cateto per la tangente dell'angolo adiacente.\par
\begin{align*}
\frac{b}{c}=&\tan\beta=\cot\gamma\\
\frac{c}{b}=&\tan\gamma=\cot\beta
\end{align*}
\section{Risoluzione triangoli rettangoli}
La risoluzione di un triangolo consiste nel trovare tutti i suoi elementi conoscendone solo alcuni. 
\begin{enumerate}
	\item Angolo acuto e ipotenusa noti
	\item Angolo acuto e cateto noti
	\item Ipotenusa e cateto
	\item Cateti noti
\end{enumerate}
I vari casi sono illustrati in seguito. 
%Ricordiamo che in un triangolo rettangolo vale il teorema di Pitagora.\index{Teorema!Pitagora} Quindi, facendo riferimento alla figura\nobs\vref{fig:TeoremaPitagora_1}:
%\begin{align*}
%a^2=&b^2+c^2\\
%a{}=&\sqrt{b^2+c^2}\\
%b^2=&a^2-c^2\\
%b{}=&\sqrt{a^2-c^2}\\
%c^2=&a^2-b^2\\
%c{}=&\sqrt{a^2-b^2}\\
%\end{align*}

%Inoltre dato che la somma degli angoli interni di un triangolo è 
%$\ang{180}$ la somma dei due angoli acuti è di $\ang{90}$. quindi
%\[\beta+\gamma=\ang{90}\]
\subsection{Angolo acuto e ipotenusa noti}
Risolviamo questo caso, conosciamo, come nella figura\nobs\vref{fig:risTriangRett_1}, l'ipotenusa $a$ e un angolo acuto, per esempio\nobs$\gamma$.
\begin{align*}
\beta=&\ang{90}-\gamma&\beta=&\ang{90}-\gamma\\
c=&a\sin\gamma&c=&a\cos\beta\\
b=&a\cos\gamma&b=&a\cos\beta
\end{align*}
\begin{esempiot}{Risolvere triangolo rettangolo}{}
Trovare gli altri elementi di un triangolo sapendo che l'ipotenusa $a=3$ e l'angolo acuto $\gamma=\ang{30}$  Otteniamo quanto segue.
\end{esempiot}
\begin{align*}
a=&3\\
\gamma=&\ang{30}\\
\beta=&\ang{90}-\ang{30}=\ang{60}\\
c=&a\sin\gamma\\
c=&3\sin\ang{30}=3\cdot\dfrac{1}{2}=\dfrac{3}{2}\\
b=&a\cos\gamma\\
b=&3\cos\ang{30}=3\cdot\dfrac{\sqrt{3}}{2}=\dfrac{3\sqrt{3}}{2}
\end{align*}
\begin{esempiot}{Numero complesso noto modulo e angolo }{}
	Un numero complesso $z$ ha modulo $\abs{z}=5$ e l'angolo $\phi=\ang{30}$.  Determinare parte reale e parte complessa.
\end{esempiot}
\begin{align*}
\abs{z}=&5\\
\phi=&\ang{30}\\
\Re\left(z\right)=&\abs{z}\cos\phi\\
\Re\left(z\right)=&5\cos(\ang{30})=5\frac{\sqrt{3}}{2}\\
\Im\left(z\right)=&\abs{z}\sin\phi\\
\Im\left(z\right)=&5\sin(\ang{30})=5\dfrac{1}{2}=\dfrac{5}{2}\\
\end{align*}
\begin{figure}
	\centering
	\includestandalone[scale=0.7]{trigonometria/pitagora_1}
	\caption{Teorema di Pitagora}
	\label{fig:TeoremaPitagora_1}
\end{figure}
\subsection{Angolo acuto e cateto noti}
Risolviamo questo caso, conosciamo, come nella figura\nobs\vref{fig:risTriangRett_2}, un cateto $b$ e un angolo acuto, ad esempio $\gamma$.
\begin{align*}
\beta=&\ang{90}-\gamma&\beta=&\ang{90}-\gamma\\
c=&b\tan\gamma&c=&a\cot\beta\\
a=&\dfrac{b}{\sin\beta}&a=&\dfrac{b}{\cos\beta}
\end{align*}
\begin{esempiot}{Risolvere triangolo rettangolo}{}
Trovare gli elementi ignoti di un triangolo rettangolo, sapendo che  il cateto $b=5$ e l'angolo $\gamma=\ang{60}$.
\end{esempiot}
\begin{align*}
b=&5\\
\gamma =&\ang{60}\\
\beta=&\ang{90}-\ang{60}=\ang{30}\\
c=&b\tan\gamma\\
c=&5\tan\ang{60}=5\cdot\sqrt{5}=5\sqrt{3}\\
a=&\dfrac{b}{\sin\beta}\\
a=&\dfrac{5}{\sin\ang{30}}=\dfrac{5}{\dfrac{1}{2}}=10
\end{align*}
\begin{esempiot}{Parte reale e $\phi$ noti}{}
La parte reale di un numero complesso è $\Re(z)=-3$. L'angolo $\phi=\ang{128}$. Trovare il modulo di $z$
\end{esempiot}
\begin{align*}
\intertext{Abbiamo vari metodi}
-3=&\abs{z}\cos(\ang{180}-\ang{128})\\
3=&\abs{z}\cos(\ang{52})\\
\abs{z}=&\dfrac{3}{\cos\ang{52}}\approx\num{4.8728}
\intertext{Metodo indiretto}
\Im(z)=&-3\tan(\ang{180}-\ang{128})\\
=&3\tan(\ang{52})\\
\approx&3.8399\\
\abs{z}\approx&\sqrt{(-3)^2+(3.8399)^2}\\
\approx&\num{4.8728}\\
\end{align*}
Una variante di quanto prima è quello che segue. Qui un angolo è noto tramite il valore di una funzione goniometrica.
\begin{esempiot}{Trovare gli elementi ignoti di un triangolo rettangolo}{}
Trovare gli elementi ignoti di un triangolo rettangolo, sapendo che  il cateto $b=4$ e l'angolo $\cos\gamma=\dfrac{4}{5}$.
\end{esempiot}
\begin{align*}
\cos\gamma=&\dfrac{4}{5}\\
b=&4
\intertext{determiniamo l'ipotenusa $a$}
b=&a\cos\gamma\\
4=&a\dfrac{4}{5}\\
\intertext{l'ipotenusa è:}
a=&\dfrac{20}{4}=5\\
\intertext{Per determinare il cateto $c$ ho bisogno di $\sin\gamma$}
\sin\gamma=&\sqrt{1-\cos^2\gamma}\\
=&\sqrt{1-\dfrac{16}{25}}\\
=&\sqrt{\dfrac{25-16}{25}}\\
=&\sqrt{\dfrac{9}{25}}=\dfrac{3}{5}\\
\intertext{ora possiamo trovare il cateto $c$}
c=&a\sin\gamma\\
c=&5\dfrac{3}{5}=3
\end{align*}
\begin{esempiot}{Numero complesso noto $\phi$ e parte reale}{}
	Di un numero complesso $z$ sono noti la fase $\phi=\ang{150}$ e la parte reale $\Re(z)=-4$. Determinare il modulo di $z$ e la parte immaginaria $\Im(z)$ 
\end{esempiot}
\begin{align*}
\phi=&\ang{150}\\
\Re(z)==&-4\\
\Im(z)=&\R(z)\tan\phi\\
\Im(z)=&-4\tan\ang{150}\\
\Im(z)=&-4\left(-\dfrac{\sqrt{3}}{3}\right)\\=&\dfrac{4}{3}\sqrt{3}\\
\abs{z}=&\sqrt{16+16\dfrac{3}{9}}\\
=&\sqrt{\dfrac{144+48}{9}}\\
=&\sqrt{\dfrac{192}{9}}=\dfrac{8\sqrt{3}}{3}
\end{align*}
\subsection{Ipotenusa e cateto}
Risolviamo questo caso, conosciamo, come nella figura\nobs\vref{fig:risTriangRett_3}, un cateto $b$ e l'ipotenusa $a$.
\begin{align*}
\beta=&\arcsin\dfrac{b}{a}&\gamma=&\arccos\dfrac{b}{a}\\
\gamma=&\ang{90}-\beta&\beta=&\ang{90}-\gamma\\
c=&a\cos\beta&c&a\sin\gamma
\end{align*}
\begin{esempiot}{Risolvere triangolo rettangolo 1}{}
Trovare gli elementi ignoti di un triangolo rettangolo sapendo che l'ipotenusa $a=2$  il cateto $b=\sqrt{5}$.
\end{esempiot}
\begin{align*}
a=&\num{2}\\
b=&\sqrt{\num{3}}\\
\beta=&\arcsin\dfrac{b}{a}\\
\beta=&\arcsin\dfrac{\sqrt{3}}{2}=\ang{60}\\
\gamma=&\ang{90}-\ang{60}=\ang{30}\\
c=&a\cos\beta\\
c=&2\cos\ang{60}=2\cdot\dfrac{1}{2}=1
\end{align*} 
\subsection{Cateti noti}
Risolviamo questo caso, conosciamo, come nella figura\nobs\vref{fig:risTriangRett_4}, con i cateti $b$ e $c$ noti.
\begin{align*}
\gamma=&\arctan\dfrac{c}{b}&\beta=&\arctan\dfrac{b}{c}\\
\beta=&\ang{90}-\gamma&\gamma=&\ang{90}-\beta\\
a=&\dfrac{c}{\sin\gamma}&a=&\dfrac{c}{\cos\beta}
\end{align*}
\begin{esempiot}{Risolvere triangolo rettangolo 2}{}
Dato il numero complesso $z=4+8\uimm$, trovarne il modulo e l'angolo $\phi$
\end{esempiot}
\begin{align*}
\Re(z)=&4\\
\Im(z)=&8\\
\abs{z}=&\sqrt{4^2+8^2}=\sqrt{80}\\
8=&4\tan\phi\\
\tan\phi=&\dfrac{8}{4}=2\\
\phi=&\arctan(2)\approxeq\ang{63.435}\\
\end{align*}
\begin{esempiot}{Numero complesso note parte reale e parte complessa}{}
	Trovare gli elementi ignoti di un triangolo rettangolo sapendo che  il cateto $b=5$ e il lato $c=12$
\end{esempiot}
\begin{align*}
b=&5\\
c=&12\\
\gamma=&\arctan\dfrac{c}{b}\\
\gamma=&\arctan\dfrac{12}{5}\approxeq\ang{67.38}\\
\beta=&\ang{90}-\gamma=\ang{67.38}\\
a=&\dfrac{c}{\sin\gamma}\\
a=&\dfrac{12}{\sin\ang{67.38}}\approxeq\dfrac{12}{\num{0.92}}\approxeq\num{13.0}
\end{align*}
\begin{figure}
	\begin{subfigure}[b]{.5\linewidth}
		\centering
\includestandalone{trigonometria/risTriangRett_1}
	\caption{Ipotenusa e angolo acuto noto}
	\label{fig:risTriangRett_1}
	\end{subfigure}%
	\begin{subfigure}[b]{.5\linewidth}
		\centering
		\includestandalone{trigonometria/risTriangRett_2}
		\caption{Cateto e angolo acuto noto}
		\label{fig:risTriangRett_2}
	\end{subfigure}
	\begin{subfigure}[b]{.5\linewidth}
		\centering
	\includestandalone{trigonometria/risTriangRett_3}
	\caption{Ipotenusa e cateto noto}
	\label{fig:risTriangRett_3}
	\end{subfigure}%
	\begin{subfigure}[b]{.5\linewidth}
		\centering
		\includestandalone{trigonometria/risTriangRett_4}
		\caption{Cateti noti}
		\label{fig:risTriangRett_4}
	\end{subfigure}
	\captionof{figure}{Risoluzione triangoli rettangoli}
	\label{fig:RisoluzioneTriangoliRettangoli}
\end{figure}
