\chapter{Funzione sinusoidale}
\label{cha:FunzioneSinusoidale}
\section{Funzioni periodiche}
\begin{definizionet}{Funzione periodica}{}
Una funzione $f$ si dice periodica di periodo $T\neq0$ se solo se \[\forall x f(x+T)=f(x)\]
\end{definizionet}\index{Funzione!periodica}
\section{Definizioni}

\begin{definizionet}{Funzione sinusoidale}{}
	Diciamo funzione sinusoidale una funzione del tipo:
	\[ y=R\sin(\omega t+\phi)\]
	\begin{itemize}
		\item $R$ è l'ampiezza positiva
		\item $\omega$ è la pulsazione positiva
		\item $\phi$ è lo sfasamento $-\pi\leq\phi\leq\pi$
	\end{itemize}
\end{definizionet}
\begin{osservazionet}{Periodicità}{}
	La funzione\[ y=R\sin(\omega t+\phi)\]è periodica di periodo\[T=\dfrac{2\pi}{\omega}\]
\end{osservazionet}
infatti\[ y=R\sin[\omega(t+T)+\phi]=R\sin[\omega(t+\dfrac{2\pi}{\omega})+\phi]=R\sin(\omega
t+2\pi+\phi)=R\sin(\omega
t+\phi)\]
Datato che il seno ha periodo $2\pi$
\begin{definizionet}{Frequenza}{}
	Se $T$ indica il periodo allora  $f=\dfrac{1}{T}$ è la frequenza.
\end{definizionet}
\begin{teoremat}{Periodo}{}
	Se una funzione $f$ è periodica di periodo $T$ allora $f(ax+b)$ $a>0$ è una funzione periodica di periodo $\dfrac{T}{a}$
\end{teoremat}
\section{Andamento funzione sinusoidale}
Consideriamo la funzione sinusoidale\[y=R\sin\omega t\] Dato che la funzione è periodica, ne tracciamo il grafico nell'intervallo \[[0\leq t\leq T]\] come nella figura\nobs\vref{fig:FunzioneSinusoidale1}

La funzione assume il suo massimo valore $R$ per $t=\dfrac{T}{4}$ mentre il suo valore minimo è per $t=\dfrac{3}{4}T$. Questo si può velocemente verificare osservando che $\sin x=1$ quando $x=\dfrac{\pi}{2}$ quindi 
\begin{align*}
\omega t=&\dfrac{\pi}{2}\\
\intertext{ma}
\omega=&\dfrac{2\pi}{T}
\intertext{quindi}
t\dfrac{2\pi}{T}=&\dfrac{\pi}{2}\\
t=&\dfrac{T}{4}
\end{align*} 
In maniera analoga si dimostra per il punto di minimo. Dal grafico è evidente che la funzione è positiva per valori di $t$ compresi tra zero e metà periodo cioè: \[[0\leq t\leq \dfrac{T}{2}]\] Mentre per \[[\dfrac{T}{2}\leq t\leq T]\] assume valori negativi. 
\begin{figure}
	\centering
	\includestandalone[width=7.5cm]{terzo/FunzSinuo/sinuo1}
	\caption{Funzione sinusoidale}
	\label{fig:FunzioneSinusoidale1}
\end{figure}
\section{Classificazioni}
I parametri che caratterizzano una funzione sinusoidale sono molti. Il confronto fra essi permette una classificazione.
\begin{definizionet}{Ampiezze diverse}{}
 Se due funzioni, come nella figura\nobs\vref{fig:FunzionegonioAmpiezzaDiv1}  \[y_1=R\sin\omega t\] \[y_2=S\sin\omega t\] hanno $R$ e $S$ diversi allora differiscono in ampiezza\index{Funzione sinusoidale!ampiezze diverse} ma hanno frequenza e periodo uguali.
\end{definizionet}
\begin{figure}
	\centering
	\includestandalone[width=7.5cm]{terzo/FunzSinuo/gonioAmpiezza1}
	\caption{Ampiezze diverse}
	\label{fig:FunzionegonioAmpiezzaDiv1}
\end{figure}
\begin{definizionet}{Frequenze diverse}{}
	Se due funzioni, come nella figura\nobs\vref{fig:FunzionegonioFrequenzaDiv1} \[y_1=R\sin\omega_1 t\] \[y_2=R\sin\omega_2 t\] hanno $\omega_1$ e $\omega_2$ diversi allora differiscono in frequenza\index{Funzione sinusoidale!frequenze diverse} quindi hanno la stessa ampiezza ma periodo e frequenza diverse.
	\begin{align*}
		T_1&=\dfrac{2\pi}{\omega_1}& T_2&=\dfrac{2\pi}{\omega_2}\\
		f_1&=\dfrac{1}{T_1}& f_2&=\dfrac{1}{T_2}\\
	\end{align*}
\end{definizionet}
\begin{figure}
	\centering
	\includestandalone[width=7.5cm]{terzo/FunzSinuo/gonioFrequenza1}
	\caption{Frequenze diverse}
	\label{fig:FunzionegonioFrequenzaDiv1}
\end{figure}
\begin{definizionet}{Anticipo e ritardo di fase}{}
	Consideriamo le funzioni \[y_1=R\sin\omega t\] e\[y_2=R\sin(\omega t +\phi)\] Queste funzioni hanno la stessa ampiezza $R$ e la stessa pulsazione $\omega$ se $\phi>0$, figura\nobs\vref{fig:FunzioneSinusoidaleAnticipo1}, allora $y_2$ è in anticipo di fase\index{Funzione sinusoidale!anticipo fase}. Invece se $\phi<0$, figura\nobs\vref{fig:FunzioneSinusoidalePosticipo1}, allora $y_2$ è in ritardo di fase\index{Funzione sinusoidale!ritardo fase}.
\end{definizionet}
\begin{figure}
	\begin{subfigure}[b]{.5\linewidth}
		\centering
		\includestandalone[width=7.5cm]{terzo/FunzSinuo/sinuo2P}
		\caption{Ritardo di fase}\label{fig:FunzioneSinusoidaleAnticipo1}
	\end{subfigure}%
	\begin{subfigure}[b]{.5\linewidth}
		\centering
		\includestandalone[width=7.5cm]{terzo/FunzSinuo/sinuo2A}
		\caption{Anticipo di fase}\label{fig:FunzioneSinusoidalePosticipo1}
	\end{subfigure}
	\captionof{figure}{Fasi}
	\label{fig:sinuoAnticipoPosticipo}
\end{figure}
\begin{definizionet}{Quadratura di fase}{}
	Consideriamo le funzioni \[y_1=R\sin\omega t\] e\[y_2=R\sin(\omega t +\phi)\]Queste funzioni sono in anticipo o in ritardo di fase, se $\phi=+\dfrac{\pi}{2}$ o  $\phi=-\dfrac{\pi}{2}$ allora $y_1$ e$y_2$ si dice che sono in quadratura\index{Funzione sinusoidale!quadratura di fase}, figura\nobs\vref{fig:sinuoQuadratura} come in una corrente bifase.
\end{definizionet}
\begin{figure}
	\begin{subfigure}[b]{.5\linewidth}
		\centering
		\includestandalone[width=7.5cm]{terzo/FunzSinuo/sinuo2QuadraturaAnticipo}
		\caption{Ritardo di fase}\label{sinuo2QuadraturaAnticipo}
	\end{subfigure}%
	\begin{subfigure}[b]{.5\linewidth}
		\centering
		\includestandalone[width=7.5cm]{terzo/FunzSinuo/sinuo2QuadraturaPosticipo}
		\caption{Anticipo di fase}\label{fig:sinuo2QuadraturaPosticipo}
	\end{subfigure}
	\captionof{figure}{Quadratura di fase}
	\label{fig:sinuoQuadratura}
\end{figure}
\begin{definizionet}{Opposizione di fase}{}
	Consideriamo le funzioni \[y_1=R\sin\omega t\] e\[y_2=R\sin(\omega t +\phi)\]Queste funzioni sono in anticipo o in ritardo di fase, se $\phi=+\pi$ o  $\phi=-\pi$ allora $y_1$ e$y_2$ si dice che sono in opposizione\index{Funzione sinusoidale!opposizione di fase}, figura\nobs\vref{fig:sinuOpposizione}.
\end{definizionet}
\begin{figure}
	\begin{subfigure}[b]{.5\linewidth}
		\centering
		\includestandalone[width=7.5cm]{terzo/FunzSinuo/sinuo2OpposizioneAnticipo}
		\caption{Ritardo di fase}\label{fig:sinuo2OpposizioneAnticipo}
	\end{subfigure}%
	\begin{subfigure}[b]{.5\linewidth}
		\centering
		\includestandalone[width=7.5cm]{terzo/FunzSinuo/sinuo2OpposizionePosticipo}
		\caption{Anticipo di fase}\label{fig:sinuo2OpposizionePosticipo}
	\end{subfigure}
	\captionof{figure}{Opposizione di fase}
	\label{fig:sinuOpposizione}
\end{figure}