\section{Triangoli qualunque}
\begin{figure}
	\centering
	\includestandalone{trigonometria/trasformazioniformula}
	\caption{Regole di trasformazione}
	\label{fig:trasformazioniformula}
\end{figure}
In un triangolo qualunque l'angolo maggiore non è necessariamente retto. La nomenclatura dei lati, spigoli e degli angoli è analoga a quella usata con un triangolo rettangolo. 

Ricordiamo che un lato è minore della somma degli altri due e che l'unione degli angoli interni è un angolo piatto. Abbiamo quindi:
 \begin{align*}
a<&b+c\\
b<&a+c\\
c<&b+a\\
\alpha+\beta+\gamma=&\ang{180}\\
\alpha+\beta+\gamma=&\frac{\pi}{2}\\
\end{align*}
La~\vref{fig:trasformazioniformula} mostra come ottenere le regole di sostituzione valide per le formule seguenti.
\begin{figure}
	\centering
	\includestandalone{trigonometria/canot_1}
	\caption{Teorema di Carnot}
	\label{fig:TeoremDiCarnot_1}
\end{figure}
\subsection{Teorema dei seni}
\begin{teoremat}{Teorema dei seni}{}
Per un triangolo qualunque\index{Teorema!seni} valgono le seguenti uguaglianze\[\dfrac{a}{\sin\alpha}=\dfrac{b}{\sin\beta}=\dfrac{c}{\sin\beta}=2R \]
\end{teoremat}
In un triangolo, il rapporto fra un suo  lato e il seno dell'angolo opposto a questo è costante ed  è uguale al diametro della circonferenza circoscritta. Segue  che il seno e l'angolo sono direttamente proporzionali. La figura\nobs\vref{fig:TeoremDeiSeni} mostra le relazioni.

Possiamo ottenere da quanto scritto in precedenza: 
\begin{align*}
a=&b\dfrac{\sin\alpha}{\sin\beta}& a=&c\dfrac{\sin\alpha}{\sin\gamma}\\
b=&a\dfrac{\sin\beta}{\sin\alpha}& b=&c\dfrac{\sin\beta}{\sin\gamma}\\
c=&a\dfrac{\sin\gamma}{\sin\alpha}& c=&b\dfrac{\sin\gamma}{\sin\beta}
\end{align*}
o queste
\begin{align*}
\sin\alpha=&\dfrac{a}{b}\sin\beta&\sin\alpha=&\dfrac{a}{c}\sin\gamma\\
\sin\beta=&\dfrac{b}{a}\sin\alpha&\sin\beta=&\dfrac{b}{c}\sin\gamma\\
\sin\gamma=&\dfrac{c}{a}\sin\alpha&\sin\gamma=&\dfrac{c}{b}\sin\beta\\
\end{align*}
\subsection{Teorema di Carnot}
Una versione più generale del teorema di Pitagora è quello di Carnot\index{Teorema!Carnot}. Partendo dalla figura\nobs\vref{fig:TeoremDiCarnot_1} avremo queste relazioni.
\begin{teoremat}{Teorema di Carnot}{}
	Per un triangolo qualunque\index{Teorema!Carnot} seguono le seguenti uguaglianze:
	\begin{align*}
	a^2=&b^2+c^2-2bc\cos\alpha\\
	b^2=&a^2+c^2-2ac\cos\beta\\
	c^2=&a^2+b^2-2ab\cos\gamma
	\end{align*}
\end{teoremat}
Riscrivendo  le precedenti equazioni, isolando i coseni, si ottiene:
\begin{align*}
\cos\alpha=&\dfrac{b^2+c^2-a^2}{2bc}\\
\cos\beta=&\dfrac{a^2+c^2-b^2}{2ac}\\
\cos\gamma=&\dfrac{a^2+b^2-c^2}{2ab}\\
\end{align*}
queste relazioni ci  permettono di trovare gli angoli:
\begin{align*}
\alpha=&\arccos(\dfrac{b^2+c^2-a^2}{2bc})\\
\beta=&\arccos(\dfrac{a^2+c^2-b^2}{2ac})\\
\gamma=&\arccos(\dfrac{a^2+b^2-c^2}{2ab})\\
\end{align*}
\begin{figure}
	\centering
	\includestandalone{trigonometria/TeoSeni_1}
	\caption{Teorema dei seni}
	\label{fig:TeoremDeiSeni}
\end{figure}
\begin{figure}
	\begin{subfigure}[b]{.5\linewidth}
		\centering
\includestandalone{trigonometria/risTriangQualunque_1}
	\caption{Un lato e due angoli noti}
	\label{fig:risTriangQqualunque_1}
	\end{subfigure}%
	\begin{subfigure}[b]{.5\linewidth}
		\centering
	\includestandalone{trigonometria/risTriangQualunque_2}
		\caption{Due lati l'angolo fra loro compreso noti}
		\label{fig:risTriangQqualunque_2}
	\end{subfigure}
	\begin{subfigure}[b]{.5\linewidth}
		\centering
		\includestandalone{trigonometria/risTriangQualunque_3}
		\caption{Due lati l'angolo fra loro non compreso noti}
		\label{fig:risTriangQqualunque_3}
	\end{subfigure}%
	\begin{subfigure}[b]{.5\linewidth}
		\centering
		\includestandalone{trigonometria/risTriangQualunque_4}
		\caption{Tre lati noti}
		\label{fig:risTriangQqualunque_4}
	\end{subfigure}
	\captionof{figure}{Risoluzione triangoli qualunque}
	\label{fig:RisoluzioneTriangoliQualunque}
\end{figure}
\section{Risoluzione di triangolo qualunque}
\subsection{Un lato e due angoli}
Se è noto il lato $c$ e gli angoli $\alpha$ e $\beta$, come nella figura\nobs\vref{fig:risTriangQqualunque_1} avremo:
\begin{align*}
\gamma=&\ang{180}-(\alpha+\beta)\\
a=&c\dfrac{\sin\alpha}{\sin\gamma}\\
b=&c\dfrac{\sin\beta}{\sin\gamma}\\
\end{align*}
\begin{esempiot}{Un lato e due angoli noti}{}
	Trovare gli altri elementi di un triangolo sapendo che il lato $a=5$ e  gli angoli $\beta=\ang{30}$ $\alpha=\ang{68}$.
\end{esempiot}
Per prima cosa partiamo dagli angoli
\begin{align*}
\beta+\alpha=&\ang{50}+\ang{68}<\ang{180}\\
\gamma=&\ang{180}-\ang{98}=\ang{82}
\end{align*}
Applicando il teorema dei seni
\begin{align*}
\dfrac{a}{\sin\alpha}=&\dfrac{b}{\sin\beta}\\
\dfrac{5}{\sin 68}=&\dfrac{b}{\sin 30}\\
b=&5\dfrac{\sin 30}{\sin 68}\approx\num[round-precision=\lungarrotandamento,round-mode=places]{2,696336857}\\
\dfrac{a}{\sin\alpha}=&\dfrac{c}{\sin\gamma}\\
\dfrac{5}{\sin 68}=&\dfrac{c}{\sin 82}\\
c=&5\dfrac{\sin 82}{\sin 68}\approx\num[round-precision=\lungarrotandamento,round-mode=places]{5,340192584}\\
\end{align*}
\subsection{Due lati e l'angolo fra essi compreso} 
In questo caso supponiamo noti i lati $b$ e $c$ e l'angolo $\alpha$ fra loro compreso. Dalla figura\nobs\vref{fig:risTriangQqualunque_2} segue:
\begin{align*}
a=&\sqrt{b^2+c^2-2bc\cos\alpha}\\
\beta=&\arccos(\dfrac{a^2+c^2-b^2}{2ac})\\
\gamma=&\arccos(\dfrac{a^2+b^2-c^2}{2ab})
\intertext{verificando che}
\alpha+&\beta+\gamma=\ang{180}
\end{align*}
\begin{esempiot}{Due lati e l'angolo fra esso compreso}{}
	Trovare gli altri elementi di un triangolo sapendo che il lato $a=40$, $b=20$ e  $\gamma=\ang{50}$.
\end{esempiot}
Applicando il teorema di Carnot
\begin{align*}
c^2=&a^2+b^2-2ab\cos\gamma\\
c^2=&20^2+40^2-2\cdot 20\cdot 40\cos 50\\
&\approx\num[round-precision=\lungarrotandamento,round-mode=places]{971.5398245}\\
c&\approx\num[round-precision=\lungarrotandamento,round-mode=places]{31.16953359}\\
\cos\alpha\approx&\dfrac{b^2+c^2-a^2}{2bc}\\
&\approx\dfrac{20^2+\num{31.1695}^2-40^2}{2\cdot 20\cdot\num{31.1695}}\\
&\approx\num[round-precision=\lungarrotandamento,round-mode=places]{-0.183241846}\\
\alpha=&\arccos(\num[round-precision=\lungarrotandamento,round-mode=places]{-0.183241846})\\
&\approx\ang[round-precision=\lungarrotandamento,round-mode=places]{100.5586455}\\
\cos\beta\approx&\dfrac{a^2+c^2-b^2}{2ac}\\
&\approx\dfrac{40^2+\num{31.1695}^2-20^2}{2\cdot 40\cdot\num{31.1695}}\\
&\approx\num[round-precision=\lungarrotandamento,round-mode=places]{0.870858423}\\
\beta=&\arccos(\num[round-precision=\lungarrotandamento,round-mode=places]{0.870858423})\\
&\approx\ang[round-precision=\lungarrotandamento,round-mode=places]{29.44145243}\\
\alpha+\beta+\gamma&\approx\ang[round-precision=\lungarrotandamento,round-mode=places]{100.5586455}+\ang[round-precision=\lungarrotandamento,round-mode=places]{29.44145243}+\ang{50}\\
\approx\ang[round-precision=\lungarrotandamento,round-mode=places]{180.0000979}\\
\end{align*}
Corretto.
\subsection{Due lati e l'angolo apposto a quello compreso}
In questo caso supponiamo che sono noti i lati $b$ e $c$ e l'angolo $\beta$ fra loro non compreso, come nella figura\nobs\vref{fig:risTriangQqualunque_3} avremo:
\begin{align*}
\intertext{verificando le soluzioni}
\gamma=&\arcsin(\dfrac{c}{b}{\sin\beta}) \\
\alpha=&\ang{180}-(\beta+\gamma)\\
a=&b\dfrac{\sin\alpha}{\sin\beta}
\end{align*}
\subsection{Tre lati}
In questo caso sono noti i lati $a$ $b$ e $c$  come nella figura\nobs\vref{fig:risTriangQqualunque_4} avremo:
\begin{align*}
\alpha=&\arccos(\dfrac{b^2+c^2-a^2}{2bc})\\
\beta=&\arccos(\dfrac{a^2+c^2-b^2}{2ac})\\
\gamma=&\arccos(\dfrac{a^2+b^2-c^2}{2ab})\\
\end{align*}
\section{Applicazioni teorema di Carnot}
\begin{figure}
	\centering
	\includestandalone[width=0.7\linewidth]{trigonometria/Parasommavettori}
	\caption{Somma di vettori}
	\label{fig:parasommavettori}
\end{figure}
\begin{teoremat}{Somme di vettori}{}
	Dati due vettori $\vec{a}$ e $\vec{b}$, se $\alpha$ è l'angolo fra essi compreso allora:
	\begin{align*}
	\abs{\overrightarrow{a+b}}=&\sqrt{b^2+a^2+2ab\cos\alpha}\\
	\abs{\overrightarrow{a-b}}=&a^2+b^2-2ab\cos\gamma
	\end{align*}
\end{teoremat}