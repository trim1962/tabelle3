\chapter{Notazione scientifica}
\section{Definizioni}
\begin{definizionet}{Notazione scientifica}{}
	Un numero è scritto in notazione scientifica\index{Notazione scientifica!definizione} se appare nella forma
	\[\text{numero}\times 10^{\text{esponente}} \]
	dove
	\begin{description}
		\item[numero] è un numero compreso tra \num{1} compreso e \num{10} escluso e non termina per zero.
		\item[esponente] un qualunque numero intero positivo o negativo
	\end{description}
\end{definizionet}
\begin{osservazionet}{Numeri in notazione scientifica}{}
	I seguenti numeri sono in notazione scientifica \numlist{3.45d-4;1.124d+6;6.23d+7;2.54d-7}
\end{osservazionet}
\begin{osservazionet}{Numeri non in notazione scientifica}{}
	I seguenti numeri non sono in notazione scientifica \numlist{34.5d-4;6.20d+7;0.254d-7}
	il primo ha il numero maggiore di 10. Il secondo il numero termina per zero. Il terzo il numero è minore di uno. 
\end{osservazionet}
\begin{esempiot}{Notazione scientifica}{}	
\begin{center}
		 \begin{tabular}{S S}
		\toprule
		\textbf{Normale} & \textbf{Notazione scientifica} \\
		\midrule
		124 & 1.24e+2 \\
	 1230 & 1.23e+3 \\
0.245 & 2.45e-1 \\
		0.0043 & 4.3d-3 \\
	 100000000 & 1.0d8\\
		\bottomrule
	\end{tabular}
\end{center}
\end{esempiot}
\begin{esempiot}{Notazione scientifica numero intero}{NSesempio1}
	Per scrivere \num{1532} in notazione scientifica, osserviamo che il numero è di quattro cifre. La virgola, posta per convenzione dopo la prima cifra, va spostata tra la cifra uno e cinque cioè verso sinistra di tre posti. Quindi\[\num{1532}=\num[scientific-notation=true]{1532} \]
\end{esempiot}\index{Notazione scientifica!numero intero}
\begin{esempiot}{Notazione scientifica numero intero}{NSesempio2}
	\num{15270} numero intero. La virgola va spostata  verso sinistra di quattro posti. Quindi\[\num{15270}=\num[scientific-notation=true]{15270}\] Lo zero non va considerato.
\end{esempiot}
\begin{esempiot}{Notazione scientifica numero decimale}{NSesempio3}
	\num{723.57} numero decimale. La virgola va spostata  verso sinistra di due posti. Quindi\[\num{723.57}=\num[scientific-notation=true]{723.57}\]
\end{esempiot}\index{Notazione scientifica!numero decimale}
\begin{esempiot}{Notazione scientifica numero decimale}{NSesempio4}
	\num{0.027} Numero decimale. La virgola va sposta dopo la prima cifra non  zero partendo da sinistra. La virgola si posiziona tra le cifre due e sette quindi si sposta verso destra di due posti, l'esponente è negativo.  Quindi\[\num{0.027}=\num[scientific-notation=true]{0.027}\]
\end{esempiot}
\begin{definizionet}{Ordine di grandezza}{}
	L'esponente della potenza del dieci viene chiamato ordine di grandezza\index{Notazione scientifica!ordine di grandezza}.
\end{definizionet}
\section{Operazioni in notazione scientifica}
\subsection{Somma differenza}
Se i numeri hanno lo stesso ordine di grandezza si procede come nel seguente esempio
\begin{esempiot}{Somma in notazione scientifica stesso ordine}{NSesempio5}
	Per sommare \num{1.24e3} con \num{3.35e3} si procede come segue
	\begin{align*}
	&\num{1.24e3}+\num{3.35e3}=
	\intertext{Si raccoglie \num{e3} e si ottinene}
	&=(\num{1.24}+\num{3.35})\num{e3}\\
	&=\num{4.59e3}
	\end{align*}
\end{esempiot}\index{Notazione scientifica!somma differenza}
Se gli ordini di grandezza sono diversi il procedimento è lievemente diverso
\begin{esempiot}{Somma in notazione scientifica ordine diverso}{NSesempio6}
	Per sommare \num{3.578e5} con \num{2.415e7} si procede come segue
	\begin{align*}
	&\num{3.578e5} + \num{2.415e7}=
	\intertext{Prima si riduce allo stesso ordine}
	&=\num{3.578e5} + \num{241.5e5}
	\intertext{Si raccoglie \num{e5} e si ottinene}
	&=(\num{3.578}+\num{241.5})\cdot\num{e5}\\
	&=\num{245.078e5}
	\intertext{correggiamo il numero e otteniamo}
	&=\num{2.45078e7}
	\end{align*}
\end{esempiot}
\subsection{Moltiplicazione divisione}
Nella moltiplicazione e nella divisione di due numeri non viene fatta la distinzione fra ordini di grandezza.
\begin{esempiot}{Moltiplicazione}{NSesempio7}
Per moltiplicare \num{3.157e5} con \num{7.12e3} si procede come segue
\begin{align*}
&\num{3.157e5} \cdot\num{7.12e3}=
\intertext{raggruppando}
&=(\num{3.157}\cdot\num{7.12})\cdot\num{e5}\cdot\num{e3}\\
&=\num{22.47784e8}
\intertext{correggiamo il numero e otteniamo}
&=\num{2.247784e9}
\end{align*}	
\end{esempiot}\index{Notazione scientifica!moltiplicazione divisione}
\begin{esempiot}{Moltiplicazione}{NSesempio8}
	Per dividere \num{3.5e5} con \num{4.2e7} si procede come segue
	\begin{align*}
	&\dfrac{\num{3.5e5}}{\num{4.2e7}}=
	\intertext{raggruppando}
	&=\dfrac{\num{3.5}}{\num{4.2}}\cdot\dfrac{\num{e5}}{\num{e7}}=\\
	&=\num{0.8333e-2}
	\intertext{correggiamo il numero e otteniamo}
	&=\num{8.333e-3}
	\end{align*}	
\end{esempiot}
\subsection{Potenze}
La potenza di un numero scritto in notazione scientifica si ottiene applicando la potenza a ciascuna delle sue parti.
 \begin{esempiot}{Potenza}{NSesempio9}
 	Per calcolare $(\num{7.38e4})^3$ si procede come segue
 	\begin{align*}
 	&(\num{7.38e4})^3=
 	\intertext{raggruppando}
 	&=(\num{7.38})^3\cdot(\num{e4})^3\\
 	&=\num{401.947272e12}
 	\intertext{correggiamo il numero e otteniamo}
 	&=\num{4.01947272e14}
 	\end{align*}	
 \end{esempiot}\index{Notazione scientifica!potenza}