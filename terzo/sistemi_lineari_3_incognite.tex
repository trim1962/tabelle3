\chapter{Sistemi lineari in tre incognite}
\label{sec:sistemiLineariIntreIncognite}


\section{Sistemi in tre incognite}
\label{sec:Sistemi3}

\begin{definizionet}{Forma normale di un sistema lineare}{}\index{Sistema!lineare!forma normale}{}
Un sistema di tre equazioni di primo grado in tre incognite $x$, $y$ e $z$, a coefficienti numerici, si dice ridotto in forma normale se è del tipo
\[\left\{
\begin{array}{*{3}{r@{~}c@{~}}r}
a_1x & + & b_1y & + & c_1z & =&d_1\\
a_2x & + & b_2y & + & c_2z & =&d_2\\
a_3x & + & b_3y & + & c_3z & =&d_3\\
\end{array}\right.\]
dove $a_1\cdots a_3\text{,~}b_1\cdots b_3\text{,~}c_1\cdots c_3$ sono i   coefficienti delle incognite, mentre $d_1\cdots d_3$ sono i termini noti.
\end{definizionet}
Vi sono vari modi per risolvere un sistema tre per tre di seguito sono esposti alcuni metodi.
\section{Metodi di risoluzione}
\label{sec:MetodiDiRisoluzionetre}
\subsection{Metodo di sostituzione}
\label{sec:Sostituzionetre}
Un sistema lineare in forma canonica, si risolve con il metodo di sostituzione\index{Sistema!metodo!sostituzione} isolando una variabile in una equazione e sostituendola nelle altre. 
\begin{esempiot}{Risolvere sistema tre per tre tramite sostituzione}{}
Risolvere il seguente sistema lineare in forma normale:
\[\left\{
\begin{array}{*{3}{r@{~}c@{~}}r}
-2x & + & y &  &  & =&4\\
-2x & & & - & z & =&-6\\
5x & &  &  &  & =&2\\
\end{array}\right.\]
\end{esempiot}
Risolviamo rispetto a $x$
\[\left\{
\begin{array}{*{3}{r@{~}c@{~}}r}
-2x & + & y &  &  & =&4\\
-2x & & & - & z & =&-6\\
 x &= & \dfrac{2}{5} &  &  & &\\
\end{array}\right.\]
Sostituiamo $x$ nelle prime due equazioni e otteniamo
\[\left\{
\begin{array}{*{3}{r@{~}c@{~}}r}
-2\cdot\dfrac{2}{5} & + & y &  &  & =&4\\[.28cm]
-2\cdot\dfrac{2}{5} & & & - & z & =&-6\\
 & &  &  & x &= &\dfrac{2}{5}\\
\end{array}\right.\left\{
\begin{array}{*{3}{r@{~}c@{~}}r}
-\dfrac{4}{5} & + & y &  &  & =&4\\[.28cm]
-\dfrac{4}{5} & & & - & z & =&-6\\
& &  &  & x &= &\dfrac{2}{5}\\
\end{array}\right.\left\{
\begin{array}{*{3}{r@{~}c@{~}}r}
 y& = & 4 & +  & \dfrac{4}{5}&\\[.28cm]
z&= &6& -&\dfrac{4}{5} &&\\
x&= &\dfrac{2}{5}  &  & &&\\
\end{array}\right.\left\{
\begin{array}{*{3}{r@{~}c@{~}}r}
y&=&\dfrac{20+4}{5} & = &\dfrac{24}{5} &\\[.28cm]
z&=&\dfrac{30-4}{5}&=&\dfrac{26}{5} & &\\
x&=&\dfrac{2}{5}&& &&\\
\end{array}\right.\]
\subsection{Metodo di riduzione}
\label{sec:Riduzionetre}
Un sistema lineare in forma canonica, si risolve con il metodo di riduzione\index{Sistema!metodo!riduzione} sommando le righe fra loro in modo che venga ridotto il numero delle incognite.
\begin{esempiot}{Risolvere sistema tre per tre tramite riduzione}{}
	Risolvere il seguente sistema lineare in forma normale:
	\[\left\{
	\begin{array}{*{3}{r@{~}c@{~}}r}
	x & + & 2y &  +&  3z& =&4\\
	2x &- & 5y& + & z & =&5\\
	4x &- & 2y & + & 2z & =&0\\
	\end{array}\right.\]
\end{esempiot}
Moltiplichiamo la prima riga per meno due e la sommiamo alla seconda quindi:
\begin{align*}
&\begin{array}{*{3}{r@{~}c@{~}}r}
-2x & - & 4y &  -&  6z& =&-4\\
2x &- & 5y& + & z & =&5\\
\midrule
 &- & 9y & - & 5z & =&-3\\
\end{array}\\
\intertext{Sostituisco alla reconda e otteniamo}
&\left\{%
\begin{array}{*{3}{r@{~}c@{~}}r}
x & + & 2y &  +&  3z& =&4\\
&- & 9y& - & 5z & =&-3\\
4x &- & 2y & + & 2z & =&0\\
\end{array}\right.%
\intertext{Moltiplichiamo la prima riga per meno quattro e la sommiamo alla terza e otteniamo:}
&\begin{array}{*{3}{r@{~}c@{~}}r}
-4x & - & 8y &  -&  12z& =&-16\\
4x &- & 2y & + & 2z & =&0\\
\midrule
&- & 10y& - & 10z & =&-16\\
\end{array}
\intertext{Che sostituiamo alla terza}
&\left\{%
\begin{array}{*{3}{r@{~}c@{~}}r}
x & + & 2y &  +&  3z& =&4\\
&- & 9y& - & 5z & =&-3\\
&- & 10y & - & 10z & =&-16\\
\end{array}\right.%
\intertext{Moltiplichiamo la seconda riga per meno due e la sommiamo alla terza e otteniamo:}
&\begin{array}{*{3}{r@{~}c@{~}}r}
& & 18y& + & 10z & =&6\\
&- & 10y & - & 10z & =&-16\\
\midrule
 & + & 8y &  & 0& =&-10\\
\end{array}\\
\intertext{Che sostuiamo alla terza}
&\left\{%
\begin{array}{*{3}{r@{~}c@{~}}r}
x & + & 2y &  +&  3z& =&4\\
&- & 9y& - & 5z & =&-3\\
& &  &  & 8y & =&-10\\
\end{array}\right.%
\intertext{Quindi}
y=&-\dfrac{5}{4}\\
\intertext{Sostituendo $y$ nella seconda, otteniamo}
-9\left(-\dfrac{5}{4}\right)-5z=&3\\
\dfrac{45}{4}-5z=&-3\\
-5z=&-\dfrac{45}{4}-3\\
5z=&\dfrac{12+45}{4}\\
z=&\dfrac{57}{20}\\
\intertext{Sostituendo $z$ e $y$ nella prima riga otteniamo}
x+2\left(-\dfrac{5}{4}\right)+3\left(\dfrac{57}{20}\right)=&4\\
x-\dfrac{10}{4}+\dfrac{171}{20}=&4\\
20x-50+171=&80\\
20x=&-41\\
x=&-\dfrac{41}{20}
\end{align*}
Quindi ricapitolando\[x=-\dfrac{41}{20}\text{,~}y=-\dfrac{5}{4}\text{,~}z=\dfrac{57}{20}\]
\begin{esempiot}{Risolvere sistema tre per tre tramite riduzione}{}
	Risolvere il seguente sistema lineare in forma normale:
	\[\left\{
	\begin{array}{*{3}{r@{~}c@{~}}r}
	x & + & y &  +&  3z& =&2\\
	2x &- & 3y& + & 4z & =&1\\
	3x &- & 2y & + & z & =&0\\
	\end{array}\right.\]
\end{esempiot}
Moltiplichiamo la prima riga per meno due e la sommiamo alla seconda, così otteniamo:
\begin{align*}
&\begin{array}{*{3}{r@{~}c@{~}}r}
-2x & - & 2y &  -&  6z& =&-4\\
2x &- & 3y& + & 4z & =&1\\
\midrule
&- & 5y & - & 2z & =&-3\\
\end{array}
\intertext{Sostituiamo alla seconda}
&\left\{%
\begin{array}{*{3}{r@{~}c@{~}}r}
x & + & y &  +&  3z& =&2\\
&- & 5y& - & 2z & =&-3\\
3x &- & 2y & + & z & =&0\\
\end{array}\right.%
\intertext{Moltiplichiamo la prima riga per meno tre e la sommiamo alla terza e otteniamo:}
&\begin{array}{*{3}{r@{~}c@{~}}r}
-3x & - & 3y &  -&  9z& =&-6\\
3x &- & 2y & + & z & =&0\\
\midrule
&- & 5y& - & 8z & =&-6\\
\end{array}\\
&\left\{%
\begin{array}{*{3}{r@{~}c@{~}}r}
x & + & y &  +&  3z& =&2\\
&- & 5y& - & 2z & =&-3\\
 &- & 5y & - & 8z & =&-6\\
\end{array}\right.%
\intertext{Moltiplichiamo la seconda riga per meno uno e la sommo alla terza e otteniamo:}
&\begin{array}{*{3}{r@{~}c@{~}}r}
& & 5y& + & 2z & =&3\\
&- & 5y & - & 8z & =&-6\\
\midrule
 &  & 0 &  -&  6z& =&-3\\
\end{array}
\intertext{Finalmente}
&\left\{%
\begin{array}{*{3}{r@{~}c@{~}}r}
x & + & y &  +&  3z& =&-2\\
&- & 5y& - & 2z & =&-3\\
& && - & 6z & =&-3\\
\end{array}\right.%
\intertext{Quindi}
z=&\dfrac{3}{6}=\dfrac{1}{2}\\
\intertext{Sostituendo $z$ nella seconda, otteniamo}
-5y -2\left(\dfrac{1}{2}\right)  =&-3\\
-5y -1  =&-3\\
-5y=&-3+1\\
-5y=&-2\\
y=&\dfrac{2}{5}\\
\intertext{Sostituendo $z$ e $y$ nella prima riga otteniamo}
x  +  \dfrac{2}{5}   +  3\left(\dfrac{1}{2}\right)=&2\\
\dfrac{10x+4+15}{10}=&\dfrac{20}{10}\\
10x+4+15=&20\\
10x=&1\\
x=&\dfrac{1}{10}
\end{align*}
Quindi ricapitolando\[x=\dfrac{1}{10}\text{,~}y=\dfrac{2}{5}\text{,~}z=\dfrac{1}{2}\]
